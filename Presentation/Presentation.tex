\documentclass{beamer}

\usepackage{subcaption}
\usepackage{graphicx}
\usepackage{tikz}
\usepackage[ngerman]{babel}
\usepackage{mathtools}
\usepackage{amsmath}
\usepackage{algorithm2e}
\usetikzlibrary{arrows}
\usetikzlibrary{shapes,snakes}

\author{Manuel Frohn}
\title{Planare Graphen - Planare Einbettung}
\institute{RWTH Aachen University, Aachen, Germany}
\date{28.08.2023}

\begin{document}
    %1
    \begin{frame}
        \maketitle
    \end{frame}

    %2
    \begin{frame}
        \frametitle{Inhaltsverzeichnis}
        \tableofcontents
    \end{frame}

    \section{Relevanz}
    %3
    \begin{frame}
        \frametitle{Relevanz}
        \begin{enumerate}
            \item Real auftretene Klasse
            \item Wichtig für Chip Design und Städteplanung
            \item Bedingung für Algorithmen und Sätze
        \end{enumerate}
    \end{frame}



    %4
    \section{Einfürung}
    \begin{frame}
        \frametitle{Planare Graphen}
        \begin{definition}
            Ein Graph G heißt planar, wenn man in der Lage
            ist, den Graphen so auf eine Ebene zu zeichnen, dass sich seine Kanten nicht
            schneiden.          
        \end{definition}
        \begin{figure}
            \begin{subfigure}{.33\textwidth}
                \centering
                \begin{tikzpicture}[every node/.style={circle, draw, fill=black!50,inner sep=0pt, minimum width=4pt},
                    edge_style/.style={draw=black}]
                    \node (v1) at (0, 0) {};
                    \node (v2) at (1, 0) {};
                    \node (v3) at (2, 0) {};
                    \node (v4) at (0, 1) {};
                    \node (v5) at (1, 1) {};
                    \node (v6) at (2, 1) {};  
                    \draw (v1) edge (v4);
                    \draw (v1) edge (v5);
                    \draw (v1) edge (v6);
                    \draw (v2) edge (v4);
                    \draw (v2) edge (v5);
                    \draw (v2) edge (v6);
                    \draw (v3) edge (v4);
                    \draw (v3) edge (v5);
                    \draw (v3) edge (v6);
                \end{tikzpicture}
                \caption{$K_{3,3}$}
            \end{subfigure}
            \begin{subfigure}{.33\textwidth}
                \centering
                \begin{tikzpicture}[every node/.style={circle, draw, fill=black!50,inner sep=0pt, minimum width=4pt},
                    edge_style/.style={draw=black}]
                    \node (v1) at (1, 0) {};
                    \node (v2) at (1, 1) {};
                    \node (v3) at (0, 2) {};
                    \node (v4) at (2, 2) {};
                    \draw (v1) edge (v2);
                    \draw (v1) edge (v3);
                    \draw (v1) edge (v4);
                    \draw (v2) edge (v3);
                    \draw (v2) edge (v4);
                    \draw (v4) edge (v3);
                \end{tikzpicture}
                \caption{$K_4$}
            \end{subfigure}
            \begin{subfigure}{.30\textwidth}
                \centering
                \begin{tikzpicture}[every node/.style={circle, draw, fill=black!50,inner sep=0pt, minimum width=4pt},
                    edge_style/.style={draw=black}]
                    \node (v1) at (0, 0) {};
                    \node (v2) at (0, 1) {};
                    \node (v3) at (1, 0) {};
                    \node (v4) at (1, 1) {};
                    \draw (v1) edge (v2);
                    \draw (v1) edge (v3);
                    \draw (v1) edge (v4);
                    \draw (v2) edge (v3);
                    \draw (v2) edge (v4);
                    \draw (v4) edge (v3);
                \end{tikzpicture}
                \caption{$K_4$}
            \end{subfigure}
        \end{figure}
    \end{frame}

    \section{Minore}
    %5
    \begin{frame}
        \frametitle{Minor}
        \begin{definition}
            M heißt Minor von G wenn M aus einem Teilgarphen
            von G, durch Kantenkontaktion hervorgeht.
        \end{definition}
        \begin{figure}
            \begin{subfigure}{.50\textwidth}
                \centering
                \begin{tikzpicture} [every node/.style={circle, draw, fill=black!50,inner sep=0pt, minimum width=4pt},
                    edge_style/.style={draw=black}]
                    \node (v1) at (0,0) {};
                    \node (v2) at (2,0) {};
                    \node (v3) at (1,2) {};
                    \draw (v1) edge (v2) {};
                    \draw (v2) edge (v3) {};
                    \draw (v3) edge (v1) {};
                \end{tikzpicture}
            \end{subfigure}
            \begin{subfigure}{.48\textwidth}
                \centering
                \begin{tikzpicture} [every node/.style={circle, draw, fill=black!50,inner sep=0pt, minimum width=4pt},
                    edge_style/.style={draw=black}]
                    \node (v1) at (0,0) {};
                    \node (v2) at (1,0) {};
                    \node (v3) at (2,0) {};
                    \node (v4) at (3,0) {};
                    \node (v5) at (2,2) {};
                    \draw (v1) edge (v2) {};
                    \draw (v2) edge (v3) {};
                    \draw (v3) edge (v4) {};
                    \draw (v4) edge (v5) {};
                    \draw (v2) edge (v5) {};
                \end{tikzpicture}
            \end{subfigure}
        \end{figure}
    \end{frame}
    %6
    \begin{frame}
        \frametitle{Kantenkontraktion}
        \centering
        \only<1>{
            \begin{tikzpicture}[every node/.style={circle, draw, fill=black!50,inner sep=0pt, minimum width=4pt},
                edge_style/.style={draw=black}]
                \node (v1) at (0, 0) {};
                \node (v2) at (1, 0) {};
                \node (v3) at (2, 0) {};
                \node (v4) at (0, 2) {};
                \node (v5) at (1, 2) {};
                \node (v6) at (2, 2) {};
                \draw (v1) edge (v2) {};
                \draw (v2) edge (v3) {};
                \draw (v2) edge (v5) {};
                \draw (v4) edge (v5) {};
                \draw (v5) edge (v6) {};
            \end{tikzpicture}
        }
        \only<2>{
            \begin{tikzpicture}[every node/.style={circle, draw, fill=black!50,inner sep=0pt, minimum width=4pt},
                edge_style/.style={draw=black}]
                \tikzset{red edge/.style={draw=red}}
                \tikzset{red node/.style={fill=red}}
                \node (v1) at (0, 0) {};
                \node [red node] (v2) at (1, 0) {};
                \node (v3) at (2, 0) {};
                \node (v4) at (0, 2) {};
                \node [red node] (v5) at (1, 2) {};
                \node (v6) at (2, 2) {};
                \draw (v1) edge (v2) {};
                \draw (v2) edge (v3) {};
                \draw[red edge] (v2) edge (v5) {};
                \draw (v4) edge (v5) {};
                \draw (v5) edge (v6) {};
            \end{tikzpicture}
        }
        \only<3>{
            \begin{tikzpicture}[every node/.style={circle, draw, fill=black!50,inner sep=0pt, minimum width=4pt},
                edge_style/.style={draw=black}]
                \tikzset{green node/.style={fill=green}}
                \node (v1) at (0, 0) {};
                \node (v3) at (2, 0) {};
                \node (v4) at (0, 2) {};
                \node (v6) at (2, 2) {};
                \node [green node] (vn) at (1, 1) {};
                \draw (v1) edge (v2) {};
                \draw (v2) edge (v3) {};
                \draw (v4) edge (v5) {};
                \draw (v5) edge (v6) {};
            \end{tikzpicture}
        }
        \only<4>{
            \begin{tikzpicture}[every node/.style={circle, draw, fill=black!50,inner sep=0pt, minimum width=4pt},
                edge_style/.style={draw=black}]
                \tikzset{green edge/.style={draw=green}}
                \node (v1) at (0, 0) {};
                \node (vn) at (1, 1) {};
                \node (v3) at (2, 0) {};
                \node (v4) at (0, 2) {};
                \node (v6) at (2, 2) {};
                \draw [green edge] (v1) edge (vn) {};
                \draw [green edge] (v3) edge (vn) {};
                \draw [green edge] (v4) edge (vn) {};
                \draw [green edge] (v6) edge (vn) {};
            \end{tikzpicture}
        }
        \only<5>{
            \begin{tikzpicture}[every node/.style={circle, draw, fill=black!50,inner sep=0pt, minimum width=4pt},
                edge_style/.style={draw=black}]
                \node (v1) at (0, 0) {};
                \node (vn) at (1, 1) {};
                \node (v3) at (2, 0) {};
                \node (v4) at (0, 2) {};
                \node (v6) at (2, 2) {};
                \draw (v1) edge (vn) {};
                \draw (v3) edge (vn) {};
                \draw (v4) edge (vn) {};
                \draw (v6) edge (vn) {};
            \end{tikzpicture}
        }
    \end{frame}

    %7
    \begin{frame}
        \frametitle{Minor Beispiel}
        \centering
        \only<1>{
            \centering
            \begin{tikzpicture} [every node/.style={circle, draw, fill=black!50,inner sep=0pt, minimum width=4pt},
                edge_style/.style={draw=black}]
                \node (v1) at (0,0) {};
                \node (v2) at (1,0) {};
                \node (v3) at (2,0) {};
                \node (v4) at (3,0) {};
                \node (v5) at (2,2) {};
                \draw (v1) edge (v2) {};
                \draw (v2) edge (v3) {};
                \draw (v3) edge (v4) {};
                \draw (v4) edge (v5) {};
                \draw (v2) edge (v5) {};
            \end{tikzpicture} 
        }
        \only<2>{
            \begin{tikzpicture} [every node/.style={circle, draw, fill=black!50,inner sep=0pt, minimum width=4pt},
                edge_style/.style={draw=black}]
                \tikzset{red edge/.style={draw=red}}
                \node (v1) at (0,0) {};
                \node (v2) at (1,0) {};
                \node (v3) at (2,0) {};
                \node (v4) at (3,0) {};
                \node (v5) at (2,2) {};
                \draw [red edge] (v1) edge (v2) {};
                \draw (v2) edge (v3) {};
                \draw (v3) edge (v4) {};
                \draw (v4) edge (v5) {};
                \draw (v2) edge (v5) {};
            \end{tikzpicture} 
        }
        \only<3>{
            \begin{tikzpicture} [every node/.style={circle, draw, fill=black!50,inner sep=0pt, minimum width=4pt},
                edge_style/.style={draw=black}]
                \tikzset{red edge/.style={draw=red}}
                \node (v2) at (1,0) {};
                \node (v3) at (2,0) {};
                \node (v4) at (3,0) {};
                \node (v5) at (2,2) {};
                \draw [red edge] (v2) edge (v3) {};
                \draw (v3) edge (v4) {};
                \draw (v4) edge (v5) {};
                \draw (v2) edge (v5) {};
            \end{tikzpicture} 
        }
        \only<4>{
            \begin{tikzpicture} [every node/.style={circle, draw, fill=black!50,inner sep=0pt, minimum width=4pt},
                edge_style/.style={draw=black}]
                \tikzset{red edge/.style={draw=red}}
                \node (v2) at (1,0) {};
                \node (v4) at (3,0) {};
                \node (v5) at (2,2) {};
                \draw (v2) edge (v4) {};
                \draw (v4) edge (v5) {};
                \draw (v2) edge (v5) {};
            \end{tikzpicture} 
        }
    \end{frame}
    \section{Wichtige Sätze}
    %8
    \begin{frame}
        \frametitle{Eulerscher Polyedersatz}
        \visible<1->{
            \begin{Satz}
                Gegeben ein planarer Graph G = (V, E) und die Anzahl seiner Gebiete $|F|$ gilt: $|V| - |E| + |F| = 2$
            \end{Satz}
        }
        \visible<2>{
            \begin{Satz}
                $G$ Planar $ \Leftrightarrow |E| \leq 3|V| - 6 \land |F| \leq 2|V| - 4$
            \end{Satz}
        }
        
    \end{frame}
    %9
    \begin{frame}
        \frametitle{Gebiete}
        \centering
        \only<1>{
            \begin{tikzpicture}[every node/.style={circle, draw, fill=black!50,inner sep=0pt, minimum width=4pt},
                edge_style/.style={draw=black}]
                \node (v1) at (0,0){};
                \node (v2) at (0,1){};
                \node (v3) at (1,0){};
                \node (v4) at (1,1){};
                \node (v5) at (2,0){};
                \node (v6) at (2,1){};
                \draw (v1) edge (v2);
                \draw (v2) edge (v4);
                \draw (v4) edge (v6);
                \draw (v5) edge (v6);
                \draw (v3) edge (v4);
                \draw (v1) edge (v3);
                \draw (v3) edge (v5);
            \end{tikzpicture}
        }
        \only<2>{
            \begin{tikzpicture}[every node/.style={circle, draw, fill=black!50,inner sep=0pt, minimum width=4pt},
                edge_style/.style={draw=black}]
                \node (v1) at (0,0){};
                \node (v2) at (0,1){};
                \node (v3) at (1,0){};
                \node (v4) at (1,1){};
                \node (v5) at (2,0){};
                \node (v6) at (2,1){};
                \draw (v1) edge [red] (v2);
                \draw (v2) edge [red] (v4);
                \draw (v4) edge (v6);
                \draw (v5) edge (v6);
                \draw (v3) edge [red] (v4);
                \draw (v1) edge [red] (v3);
                \draw (v3) edge (v5);
            \end{tikzpicture}
        }
        \only<3>{
            \begin{tikzpicture}[every node/.style={circle, draw, fill=black!50,inner sep=0pt, minimum width=4pt},
                edge_style/.style={draw=black}]
                \node (v1) at (0,0){};
                \node (v2) at (0,1){};
                \node (v3) at (1,0){};
                \node (v4) at (1,1){};
                \node (v5) at (2,0){};
                \node (v6) at (2,1){};
                \draw (v1) edge (v2);
                \draw (v2) edge (v4);
                \draw (v4) edge [blue](v6);
                \draw (v5) edge [blue](v6);
                \draw (v3) edge [blue] (v4);
                \draw (v1) edge (v3);
                \draw (v3) edge [blue](v5);
            \end{tikzpicture}
        }
        \only<4>{
            \begin{tikzpicture}[every node/.style={circle, draw, fill=black!50,inner sep=0pt, minimum width=4pt},
                edge_style/.style={draw=black}]
                \node (v1) at (0,0){};
                \node (v2) at (0,1){};
                \node (v3) at (1,0){};
                \node (v4) at (1,1){};
                \node (v5) at (2,0){};
                \node (v6) at (2,1){};
                \draw (v1) edge [green](v2);
                \draw (v2) edge [green](v4);
                \draw (v4) edge [green](v6);
                \draw (v5) edge [green](v6);
                \draw (v3) edge (v4);
                \draw (v1) edge [green](v3);
                \draw (v3) edge [green](v5);
            \end{tikzpicture}
        }
    \end{frame}
    %10
    \begin{frame}
        \frametitle{Satz von Kuratowski}
        \begin{Satz}
            Ein Graph ist genau dann planar, wenn er weder den $K{3,3}$ noch den $K_5$ als Minor enthält
        \end{Satz}
        \begin{figure}
            \begin{subfigure}{.50\textwidth}
                \centering
                \begin{tikzpicture}[every node/.style={circle, draw, fill=black!50,inner sep=0pt, minimum width=4pt},
                    edge_style/.style={draw=black}]
                    \node (v1) at (0, 0) {};
                    \node (v2) at (1, 0) {};
                    \node (v3) at (2, 0) {};
                    \node (v4) at (0, 1) {};
                    \node (v5) at (1, 1) {};
                    \node (v6) at (2, 1) {};  
                    \draw (v1) edge (v4);
                    \draw (v1) edge (v5);
                    \draw (v1) edge (v6);
                    \draw (v2) edge (v4);
                    \draw (v2) edge (v5);
                    \draw (v2) edge (v6);
                    \draw (v3) edge (v4);
                    \draw (v3) edge (v5);
                    \draw (v3) edge (v6);
                \end{tikzpicture}
                \caption{$K_{3,3}$}
            \end{subfigure}
            \begin{subfigure}{.48\textwidth}
                \centering
                \begin{tikzpicture}[every node/.style={circle, draw, fill=black!50,inner sep=0pt, minimum width=4pt},
                    edge_style/.style={draw=black}]
                    \node (v1) at (1, 0) {};
                    \node (v2) at (3, 0) {};
                    \node (v3) at (0, 2) {};
                    \node (v4) at (4, 2) {};
                    \node (v5) at (2, 3) {};
                    \draw (v1) edge (v2);
                    \draw (v1) edge (v3);
                    \draw (v1) edge (v4);
                    \draw (v1) edge (v5);
                    \draw (v2) edge (v3);
                    \draw (v2) edge (v4);
                    \draw (v2) edge (v5);
                    \draw (v3) edge (v4);
                    \draw (v3) edge (v5);
                    \draw (v4) edge (v5);
                \end{tikzpicture}
                \caption{$K_5$}
            \end{subfigure}
        \end{figure}
    \end{frame}
    \section{Komponenten, Separatoren und Bikomponenten}
    %11
    \begin{frame}
        \frametitle{Komponente}
        \centering
        \begin{Definition}
            Ein maximale Teilgraph $G' = (V',G') \subset G$ mit $\forall v \in V' \forall w \in V': v \xRightarrow{*} w$ heißt Komponente von G
        \end{Definition}
        \begin{tikzpicture}[every node/.style={circle, draw, fill=black!50,inner sep=0pt, minimum width=4pt},
            edge_style/.style={draw=black}]
            \node (v1) at (0,0) {};
            \node (v2) at (0,1) {};
            \node (v3) at (1,0) {};
            \draw (v1) edge (v2) {};
            \draw (v1) edge (v3) {};

            \node (v4) at (4,0){};
            \node (v5) at (4,1){};
            \draw (v4) edge (v5){};
        \end{tikzpicture}
    \end{frame}

    %12
    \begin{frame}
        \frametitle{Separator}
        \begin{Definition}
            Gegeben ein Graph $G=(V,E)$ heißt ein Knoten $u \in V$ Separator, wenn für alle Pfade $p = v \xRightarrow{*} w$ mit $v,w \in V$
            gilt: $u \in p$
        \end{Definition}
        \centering
        \only<2>{
            \begin{tikzpicture}[every node/.style={circle, draw, fill=black!50,inner sep=0pt, minimum width=4pt},
                edge_style/.style={draw=black}]
                \node (v1) at (0,1){};
                \node (v2) at (1,2){};
                \node (v3) at (1,0){};
                \node (v4) at (2,1){};
                \node (v5) at (3,2){};
                \node (v6) at (3,0){};
                \node (v7) at (4,1){}; 
                \draw (v1) edge (v2){};
                \draw (v1) edge (v3){};
                \draw (v2) edge (v4){};
                \draw (v3) edge (v4){};
                \draw (v4) edge (v5){};
                \draw (v4) edge (v6){};
                \draw (v5) edge (v7){};
                \draw (v6) edge (v7){};
            \end{tikzpicture}
        }
        \only<3>{
            \begin{tikzpicture}[every node/.style={circle, draw, fill=black!50,inner sep=0pt, minimum width=4pt},
                edge_style/.style={draw=black}]
                \tikzset{red node/.style={fill=red}}
                \node (v1) at (0,1){};
                \node (v2) at (1,2){};
                \node (v3) at (1,0){};
                \node (v4)[red node]  at (2,1){};
                \node (v5) at (3,2){};
                \node (v6) at (3,0){};
                \node (v7) at (4,1){}; 
                \draw (v1) edge (v2){};
                \draw (v1) edge (v3){};
                \draw (v2) edge (v4){};
                \draw (v3) edge (v4){};
                \draw (v4) edge (v5){};
                \draw (v4) edge (v6){};
                \draw (v5) edge (v7){};
                \draw (v6) edge (v7){};
            \end{tikzpicture}
        }
        \only<4>{
            \begin{tikzpicture}[every node/.style={circle, draw, fill=black!50,inner sep=0pt, minimum width=4pt},
                edge_style/.style={draw=black}]
                \tikzset{red node/.style={fill=red}}
                \node (v1) at (0,1){};
                \node (v2) at (1,2){};
                \node (v3) at (1,0){};
                \node (v5) at (3,2){};
                \node (v6) at (3,0){};
                \node (v7) at (4,1){}; 
                \draw (v1) edge (v2){};
                \draw (v1) edge (v3){};
                \draw (v5) edge (v7){};
                \draw (v6) edge (v7){};
            \end{tikzpicture}
        }
    \end{frame}

    %13
    \begin{frame}
        \frametitle{Bikomponente}
        \begin{Definition}
            Ein maximale Teilgraph $G' = (V',G') \subset G$ ohne Separatoren heißt Bi-Komponente von G
        \end{Definition}
        \centering
        \only<1>{
            \begin{tikzpicture}[every node/.style={circle, draw, fill=black!50,inner sep=0pt, minimum width=4pt},
                edge_style/.style={draw=black}]
                \node (v1) at (0,1){};
                \node (v2) at (1,2){};
                \node (v3) at (1,0){};
                \node (v4) at (2,1){};
                \node (v5) at (3,2){};
                \node (v6) at (3,0){};
                \node (v7) at (4,1){}; 
                \draw (v1) edge (v2){};
                \draw (v1) edge (v3){};
                \draw (v2) edge (v4){};
                \draw (v3) edge (v4){};
                \draw (v4) edge (v5){};
                \draw (v4) edge (v6){};
                \draw (v5) edge (v7){};
                \draw (v6) edge (v7){};
            \end{tikzpicture}
        }
        \only<2>{
            \begin{tikzpicture}[every node/.style={circle, draw, fill=black!50,inner sep=0pt, minimum width=4pt},
                edge_style/.style={draw=black}]
                \tikzset{red edge/.style={draw=red}}
                \tikzset{red node/.style={fill=red}}
                \node (v1) at (0,1){};
                \node (v2) at (1,2){};
                \node (v3) at (1,0){};
                \node (v4)[red node] at (2,1){};
                \node (v5) at (3,2){};
                \node (v6) at (3,0){};
                \node (v7) at (4,1){}; 
                \draw (v1) edge (v2){};
                \draw (v1) edge (v3){};
                \draw (v2) edge (v4){};
                \draw (v3) edge (v4){};
                \draw (v4) edge (v5){};
                \draw (v4) edge (v6){};
                \draw (v5) edge (v7){};
                \draw (v6) edge (v7){};
            \end{tikzpicture}
        }
        \only<3>{
            \begin{tikzpicture}[every node/.style={circle, draw, fill=black!50,inner sep=0pt, minimum width=4pt},
                edge_style/.style={draw=black}]
                \tikzset{green edge/.style={draw=green}}
                \tikzset{blue edge/.style={draw=blue}}
                \tikzset{red node/.style={fill=red}}
                \tikzset{green node/.style={fill=green}}
                \tikzset{blue node/.style={fill=blue}}
                \node (v1)[green node] at (0,1){};
                \node (v2)[green node] at (1,2){};
                \node (v3)[green node] at (1,0){};
                \node (v4)[red node] at (2,1){};
                \node (v5) at (3,2){};
                \node (v6) at (3,0){};
                \node (v7) at (4,1){}; 
                \draw (v1)[green edge] edge (v2){};
                \draw (v1)[green edge] edge (v3){};
                \draw (v2)[green edge] edge (v4){};
                \draw (v3)[green edge] edge (v4){};
                \draw (v4) edge (v5){};
                \draw (v4) edge (v6){};
                \draw (v5) edge (v7){};
                \draw (v6) edge (v7){};
            \end{tikzpicture}
        }
        \only<4>{
            \begin{tikzpicture}[every node/.style={circle, draw, fill=black!50,inner sep=0pt, minimum width=4pt},
                edge_style/.style={draw=black}]
                \tikzset{green edge/.style={draw=green}}
                \tikzset{blue edge/.style={draw=blue}}
                \tikzset{red node/.style={fill=red}}
                \tikzset{green node/.style={fill=green}}
                \tikzset{blue node/.style={fill=blue}}
                \node (v1)[green node] at (0,1){};
                \node (v2)[green node] at (1,2){};
                \node (v3)[green node] at (1,0){};
                \node (v4)[red node] at (2,1){};
                \node (v5)[blue node] at (3,2){};
                \node (v6)[blue node] at (3,0){};
                \node (v7)[blue node] at (4,1){}; 
                \draw (v1)[green edge] edge (v2){};
                \draw (v1)[green edge] edge (v3){};
                \draw (v2)[green edge] edge (v4){};
                \draw (v3)[green edge] edge (v4){};
                \draw (v4)[blue edge] edge (v5){};
                \draw (v4)[blue edge] edge (v6){};
                \draw (v5)[blue edge] edge (v7){};
                \draw (v6)[blue edge] edge (v7){};
            \end{tikzpicture}
        }
    \end{frame}
    %15
    \section{Der Einbettungsalgorithmus}
    \begin{frame}
        \frametitle{Bikomponenten und Planarität}
        \begin{Satz}
            Ein Graph ist genau dan Planar, wenn seine Bikomponenten planar sind
        \end{Satz}
        \begin{figure}
            \centering
            \begin{subfigure}{.50\textwidth}
                \centering
                \begin{tikzpicture}[every node/.style={circle, draw, fill=black!50,inner sep=0pt, minimum width=4pt},
                    edge_style/.style={draw=black}]
                    \node (v1) at (0,0){};
                    \node (v2) at (0,2){};
                    \node (v3) at (2,1){};
                    \node (v4) at (1,1){};
                    \draw (v1)[blue] edge (v2){};
                    \draw (v1)[blue] edge (v3){};
                    \draw (v2)[blue] edge (v3){};
                    \draw (v3)[green] edge (v4){};
                \end{tikzpicture}
            \end{subfigure}
            \begin{subfigure}{.48\textwidth}
                \centering
                \begin{tikzpicture}[every node/.style={circle, draw, fill=black!50,inner sep=0pt, minimum width=4pt},
                    edge_style/.style={draw=black}]
                    \node (v1) at (0,0){};
                    \node (v2) at (0,2){};
                    \node (v3) at (2,1){};
                    \node (v4) at (3,1){};
                    \draw (v1)[blue] edge (v2){};
                    \draw (v1)[blue] edge (v3){};
                    \draw (v2)[blue] edge (v3){};
                    \draw (v3)[green] edge (v4){};
                \end{tikzpicture}
            \end{subfigure}
        \end{figure}
    
    \end{frame}

    \begin{frame}
        \frametitle{Knoten in der Planaren Einbettung}
        \centering
        \only<1>{
            \begin{tikzpicture}[every node/.style={circle, draw, fill=black!50,inner sep=0pt, minimum width=4pt},
                edge_style/.style={draw=black}]
                \node (v1) at (0,0){};
                \node (v2) at (2,0){};
                \node (v3)[red] at (1,1){};
                \node (v4) at (0,2){};
                \node (v5) at (2,2){};
                \draw (v1) edge [blue] (v2){};
                \draw (v1) edge [blue] (v3){};
                \draw (v2) edge [blue] (v3){};
                \draw (v3) edge [green] (v5){};
                \draw (v3) edge [green] (v4){};
                \draw (v4) edge [green] (v5){};
            \end{tikzpicture}
        }
        \only<2>{
            \begin{figure}
                \begin{subfigure}{.50\textwidth}
                    \centering
                    \begin{tikzpicture}[every node/.style={circle, draw, fill=black!50,inner sep=0pt, minimum width=4pt},
                        edge_style/.style={draw=black}]
                        \node (v1) at (0,0){1};
                        \node (v2) at (2,0){2};
                        \node (v3)[fill=red] at (1,1){3};
                        \node (v4) at (0,2){4};
                        \node (v5) at (2,2){5};
                        \draw (v1) edge [blue] (v2){};
                        \draw (v1) edge [blue] (v3){};
                        \draw (v2) edge [blue] (v3){};
                        \draw (v3) edge [green] (v5){};
                        \draw (v3) edge [green] (v4){};
                        \draw (v4) edge [green] (v5){};
                    \end{tikzpicture}    
                \end{subfigure}
                \begin{subfigure}{.48\textwidth}
                    \centering
                    \begin{tikzpicture}[every node/.style={circle, draw, fill=black!50,inner sep=0pt, minimum width=4pt},
                        edge_style/.style={draw=black}]
                        \node (v1)[rectangle] at (1,0){3};
                        \node (v2) at (0,0){(3,2)};
                        \node (v3) at (2,0){(3,5)};
                        \node (v4) at (0,1){(3,4)};
                        \node (v5) at (2,1){(3,1)};
                        \draw (v1) edge (v2){};
                        \draw (v1) edge (v3){};
                        \draw (v3) edge (v5){};
                        \draw (v5) edge (v4){};
                        \draw (v4) edge (v2){};
                    \end{tikzpicture}
                \end{subfigure}
            \end{figure}
        }
    
    \end{frame}

    \begin{frame}
        \frametitle{Der Einbettungsalgorithmus - Idee}
        Bette Kante für Kante ein, so dass jede Teileinbettung planar ist
        \underbar{Frage 1:} Wo ?\\
        \underbar{Frage 2:} Wann ?\\

        \begin{figure}
            \begin{subfigure}{.50\textwidth}
                \centering
                
                
                \begin{tikzpicture}[every node/.style={circle, draw, fill=black!50,inner sep=0pt, minimum width=4pt},
                    edge_style/.style={draw=black}]
                    \node (v1) at (0,0){};
                    \node (v2) at (1,0){};
                    \node (v3) at (0,1){};
                    \node (v4) at (1,1){};
                    \draw (v1) edge (v2){};
                    \draw (v2) edge (v4){};
                    \draw (v4) edge (v3){};
                    \draw (v1) edge (v3){};
                    \draw (v1) edge (v4){};
                    \draw (v2) edge [red]  (v3){};
                \end{tikzpicture}
            \end{subfigure}
        \end{figure}


        
    \end{frame}

    \begin{frame}
        \frametitle{Wo ?}
        \centering
        \only<1>{
            \begin{tikzpicture}[every node/.style={circle, draw, fill=black!50,inner sep=0pt, minimum width=4pt},
                edge_style/.style={draw=black}]
                \node (v1) at (0,0){};
                \node (v2) at (1,0){};
                \node (v3) at (2,0){};
                \node (v4) at (1,1){};
                \node (v5) at (1,2){};
                \draw (v1) edge (v2){};
                \draw (v2) edge (v3){};
                \draw (v1) edge (v5){};
                \draw (v3) edge (v5){};
                \draw (v2) edge (v4){};
                \draw (v4) edge (v5){};
            \end{tikzpicture}
        }
        \only<2>{
            \begin{tikzpicture}[every node/.style={circle, draw, fill=black!50,inner sep=0pt, minimum width=4pt},
                edge_style/.style={draw=black}]
                \node (v1) at (0,0){};
                \node (v2) at (1,0){};
                \node (v3) at (2,0){};
                \node (v4) at (1,1){};
                \node (v5) at (1,2){};
                \draw (v1) edge (v2){};
                \draw (v2) edge (v3){};
                \draw (v1) edge (v5){};
                \draw (v3) edge (v5){};
                \draw (v2) edge (v4){};
                \draw (v4) edge (v5){};
                \draw (v1) edge[red, bend left] (v3){};
            \end{tikzpicture}
        }
        \only<3>{
            \begin{tikzpicture}[every node/.style={circle, draw, fill=black!50,inner sep=0pt, minimum width=4pt},
                edge_style/.style={draw=black}]
                \node (v1) at (0,0){};
                \node (v2) at (1,0){};
                \node (v3) at (2,0){};
                \node (v4) at (1,1){};
                \node (v5) at (1,2){};
                \draw (v1) edge (v2){};
                \draw (v2) edge (v3){};
                \draw (v1) edge (v5){};
                \draw (v3) edge (v5){};
                \draw (v2) edge (v4){};
                \draw (v4) edge (v5){};
                \draw (v1) edge[red, bend right] (v3){};
            \end{tikzpicture}
        }
    \end{frame}

    \begin{frame}
        \frametitle{Die Datenstruktur}
        \only<1>{
            \centering
            \begin{tikzpicture}[every node/.style={circle, draw, fill=black!10,inner sep=0pt, minimum width=9pt},
                edge_style/.style={draw=black}]
                \tikzset{activ/.style={rectangle}}
                \tikzset{red node/.style={fill=red}}

                \node (v0) at (0,0){0'};
                \node (v1) at (0,1){1};
                \draw (v0) edge (v1){};
                \node (v1') at (0,2){1'};
                \node (v2) at (0,3){2};
                \node (v4) at (0,4) {4};
                \node (v3) at (-1,3){3};
                \draw (v1') edge (v2){};
                \draw (v1') edge (v3){};
                \draw (v2) edge (v4){};
                \draw (v1') edge [bend right] (v4){};
                \draw (v3) edge [bend left] (v4){};


            \end{tikzpicture}
        }
        \only<2>{
            \begin{figure}
                \begin{subfigure}{.5\textwidth}
                    \begin{tikzpicture}[every node/.style={circle, draw, fill=black!10,inner sep=0pt, minimum width=9pt},
                        edge_style/.style={draw=black}]
                        \tikzset{activ/.style={rectangle}}
                        \tikzset{red node/.style={fill=red}}
        
                        \node (v0) at (0,0){0'};
                        \node (v1) at (0,1){1};
                        \draw (v0) edge (v1){};
                        \node (v1') at (0,2){1'};
                        \node (v2) at (0,3){2};
                        \node (v4) at (0,4) {4};
                        \node (v3) at (-1,3){3};
                        \draw (v1') edge (v2){};
                        \draw (v1') edge (v3){};
                        \draw (v2) edge (v4){};
                        \draw (v1') edge [bend right] (v4){};
                        \draw (v3) edge [bend left] (v4){};
        
        
                    \end{tikzpicture}
                \end{subfigure}
                \begin{subfigure}{.48\textwidth}
                    \begin{tikzpicture}[every node/.style={draw, circle, fill=black!10,inner sep=0pt, minimum width=9pt},
                        edge_style/.style={draw=black}]
                        \tikzset{vertex/.style={rectangle}}
                        \tikzset{down/.style = {->,> = latex'}}

                        \node (v0) [vertex] at (0,0){0};
                        \node (v1) [vertex] at (0,1){1};
                        \node (v1')[vertex] at (0,2) {1};
                        \node (v2) [vertex] at (0,4) {2};
                        \node (v4) [vertex] at (0,6){4};
                        \node (v3) [vertex] at (-3,4){3};
                       
                        
                    \end{tikzpicture}
                \end{subfigure}
            \end{figure}
        }
        \only<3>{
            \begin{figure}
                \begin{subfigure}{.5\textwidth}
                    \begin{tikzpicture}[every node/.style={circle, draw, fill=black!10,inner sep=0pt, minimum width=9pt},
                        edge_style/.style={draw=black}]
                        \tikzset{activ/.style={rectangle}}
                        \tikzset{red node/.style={fill=red}}
        
                        \node (v0) at (0,0){0'};
                        \node (v1) at (0,1){1};
                        \draw (v0) edge (v1){};
                        \node (v1') at (0,2){1'};
                        \node (v2) at (0,3){2};
                        \node (v4) at (0,4) {4};
                        \node (v3) at (-1,3){3};
                        \draw (v1') edge (v2){};
                        \draw (v1') edge (v3){};
                        \draw (v2) edge (v4){};
                        \draw (v1') edge [bend right] (v4){};
                        \draw (v3) edge [bend left] (v4){};
        
        
                    \end{tikzpicture}
                \end{subfigure}
                \begin{subfigure}{.48\textwidth}
                    \begin{tikzpicture}[every node/.style={draw, circle, fill=black!10,inner sep=0pt, minimum width=9pt},
                        edge_style/.style={draw=black}]
                        \tikzset{vertex/.style={rectangle}}
                        \tikzset{down/.style = {->,> = latex'}}

                        \node (v0) [vertex] at (0,0){0};
                        \node (v1) [vertex] at (0,1){1};
                        \node (v1')[vertex] at (0,2) {1};
                        \draw (v1')[down, red] edge (v1){};
                        \node (v2) [vertex] at (0,4) {2};
                        \node (v4) [vertex] at (0,6){4};
                        \node (v3) [vertex] at (-3,4){3};
                    \end{tikzpicture}
                \end{subfigure}
            \end{figure}
        }
        \only<4>{
            \begin{figure}
                \begin{subfigure}{.5\textwidth}
                    \begin{tikzpicture}[every node/.style={circle, draw, fill=black!10,inner sep=0pt, minimum width=9pt},
                        edge_style/.style={draw=black}]
                        \tikzset{activ/.style={rectangle}}
                        \tikzset{red node/.style={fill=red}}
        
                        \node (v0) at (0,0){0'};
                        \node (v1) at (0,1){1};
                        \draw (v0) edge (v1){};
                        \node (v1') at (0,2){1'};
                        \node (v2) at (0,3){2};
                        \node (v4) at (0,4) {4};
                        \node (v3) at (-1,3){3};
                        \draw (v1') edge (v2){};
                        \draw (v1') edge (v3){};
                        \draw (v2) edge (v4){};
                        \draw (v1') edge [bend right] (v4){};
                        \draw (v3) edge [bend left] (v4){};
        
        
                    \end{tikzpicture}
                \end{subfigure}
                \begin{subfigure}{.48\textwidth}
                    \begin{tikzpicture}[every node/.style={draw, circle, fill=black!10,inner sep=0pt, minimum width=9pt},
                        edge_style/.style={draw=black}]
                        \tikzset{vertex/.style={rectangle}}
                        \tikzset{down/.style = {->,> = latex'}}

                        \node (v0) [vertex] at (0,0){0};
                        \node (v1) [vertex] at (0,1){1};
                        \node (e1)  at (1,0) {(0,1)};
                        \node (e2)  at (1,1) {(1,0)};
                        \draw (e1) [red, thick] edge (e2){};

                        \node (v1')[vertex] at (0,2) {1};
                        \draw (v1')[down, red] edge (v1){};
                        \node (e3) at (1,2) {(1,4)};
                        \node (e4) at (2,2) {(4,1)};
                        \draw (e3) [red, thick] edge (e4){};
                        \node (e5) at (-1,2) {(1,3)};
                        \node (e6) at (-2,2) {(3,1)};
                        \draw (e5) [red, thick] edge (e6){};
                        \node (e7) at (0, 3) {(1,2)};
                        \node (e8) at (1, 3) {(2,1)};
                        \draw (e7) [red, thick] edge (e8){};
                        \node (v2) [vertex] at (0,4) {2};
                        \node (e9) at (0, 5) {(2,4)};
                        \node (e10) at (-1, 5) {(4,2)};
                        \draw (e9) [red, thick] edge (e10){};
                        \node (v4) [vertex] at (0,6){4};
                        \node (v3) [vertex] at (-3,4){3};
                        \node (e11) at (-3, 5){(3,4)};
                        \node (e12) at (-3, 6){(4,3)};
                        \draw (e11) [red, thick] edge (e12){};

                    \end{tikzpicture}
                \end{subfigure}
            \end{figure}
        }
        \only<5>{
            \begin{figure}
                \begin{subfigure}{.5\textwidth}
                    \begin{tikzpicture}[every node/.style={circle, draw, fill=black!10,inner sep=0pt, minimum width=9pt},
                        edge_style/.style={draw=black}]
                        \tikzset{activ/.style={rectangle}}
                        \tikzset{red node/.style={fill=red}}
        
                        \node (v0) at (0,0){0'};
                        \node (v1) at (0,1){1};
                        \draw (v0) edge (v1){};
                        \node (v1') at (0,2){1'};
                        \node (v2) at (0,3){2};
                        \node (v4) at (0,4) {4};
                        \node (v3) at (-1,3){3};
                        \draw (v1') edge (v2){};
                        \draw (v1') edge (v3){};
                        \draw (v2) edge (v4){};
                        \draw (v1') edge [bend right] (v4){};
                        \draw (v3) edge [bend left] (v4){};
        
        
                    \end{tikzpicture}
                \end{subfigure}
                \begin{subfigure}{.48\textwidth}
                    \begin{tikzpicture}[every node/.style={draw, circle, fill=black!10,inner sep=0pt, minimum width=9pt},
                        edge_style/.style={draw=black}]
                        \tikzset{vertex/.style={rectangle}}
                        \tikzset{down/.style = {->,> = latex'}}

                        \node (v0) [vertex] at (0,0){0};
                        \node (v1) [vertex] at (0,1){1};
                        \node (e1)  at (1,0) {(0,1)};
                        \node (e2)  at (1,1) {(1,0)};
                        \draw (e1) [red, thick] edge (e2){};
                        \draw (v0) edge (e1){};
                        \draw (v1) edge (e2){};
                        \draw (v0)[bend right] edge (e1){};
                        \draw (v1)[bend left] edge (e2){};
                        \node (v1')[vertex] at (0,2) {1};
                        \draw (v1')[down, red] edge (v1){};
                        \node (e3) at (1,2) {(1,4)};
                        \node (e4) at (2,2) {(4,1)};
                        \draw (e3) [red, thick] edge (e4){};
                        \node (e5) at (-1,2) {(1,3)};
                        \node (e6) at (-2,2) {(3,1)};
                        \draw (e5) [red, thick] edge (e6){};
                        \node (e7) at (0, 3) {(1,2)};
                        \node (e8) at (1, 3) {(2,1)};
                        \draw (e7) [red, thick] edge (e8){};
                        \node (v2) [vertex] at (0,4) {2};
                        \node (e9) at (0, 5) {(2,4)};
                        \node (e10) at (-1, 5) {(4,2)};
                        \draw (e9) [red, thick] edge (e10){};
                        \draw (v1') edge (e3){};
                        \draw (v1') edge (e5){};
                        \draw (e3) edge (e7){};
                        \draw (e7) edge (e5){};
                        \node (v4) [vertex] at (0,6){4};
                        \node (v3) [vertex] at (-3,4){3};
                        \node (e11) at (-3, 5){(3,4)};
                        \node (e12) at (-3, 6){(4,3)};
                        \draw (e11) [red, thick] edge (e12){};
                        \draw (e6) edge (v3){};
                        \draw (v3) edge (e11){};
                        \draw (e6) edge (e11){};
                        \draw (e8) edge (v2){};
                        \draw (v2) edge (e9){};
                        \draw (e8) edge (e9){};
                        \draw (e12) edge (v4){};
                        \draw (e12) edge (e10){};
                        \draw (v4)[bend left] edge (e4){};
                        \draw (e10)[bend left] edge (e4){};

                    \end{tikzpicture}
                \end{subfigure}
            \end{figure}
        }
            
    \end{frame}

    \begin{frame}
        \frametitle{Der Einbettungsalgorithmus}
        \begin{enumerate}
            \item Erstelle einen Palmtree des Eingabegraphens $G$
            \item Für alle Knoten $v$ des Graphens, in inverser DFI Ordnung:
            \item \; Für alle Kanten $e$, die $v$ mit einem DFS Nachkommen verbinden:
            \item \; \; Führe $WalkUp(e)$ aus
            \item \; Führe auf den Graphen, der durch die Pfade aus Schritt 4 induziert wird, $WalkDown$ aus  
        \end{enumerate}
    \end{frame}

    \begin{frame}
        \frametitle{Palmtree}
        \centering
        \only<1>{
            \begin{tikzpicture}[every node/.style={circle, draw, fill=black!50,inner sep=0pt, minimum width=4pt},
                edge_style/.style={draw=black}]
                \node (v1) at (1, 0) {};
                \node (v2) at (0, 1) {};
                \node (v3) at (0, 2) {};
                \node (v4) at (1, 3) {};
                \node (v5) at (2, 2) {};
                \node (v6) at (2, 1) {};
                \draw (v1) edge (v2) {};
                \draw (v1) edge (v5) {};
                \draw (v1) edge (v4) {};
                \draw (v1) edge (v3) {};
                \draw (v1) edge (v6) {};
                \draw (v2) edge (v3) {};
                \draw (v2) edge (v4) {};
                \draw (v2) edge (v5) {};
                \draw (v3) edge (v4) {};
                \draw (v4) edge (v5) {};
                \draw (v5) edge (v6) {};
            \end{tikzpicture}
        }
        \only<2>{
            \begin{figure}
                \centering
                \begin{subfigure}{.5\textwidth}
                    \centering
                    \begin{tikzpicture}[every node/.style={circle, draw, fill=black!50,inner sep=0pt, minimum width=4pt},
                        edge_style/.style={draw=black}]
                        \tikzset{red edge/.style={draw=red}}
                        \tikzset{red node/.style={fill=red}}
                        \node (v1)[red node] at (1, 0) {0};
                        \node (v2) at (0, 1) {};
                        \node (v3) at (0, 2) {};
                        \node (v4) at (1, 3) {};
                        \node (v5) at (2, 2) {};
                        \node (v6) at (2, 1) {};
                        \draw (v1) edge (v2) {};
                        \draw (v1) edge (v5) {};
                        \draw (v1) edge (v4) {};
                        \draw (v1) edge (v3) {};
                        \draw (v1) edge (v6) {};
                        \draw (v2) edge (v3) {};
                        \draw (v2) edge (v4) {};
                        \draw (v2) edge (v5) {};
                        \draw (v3) edge (v4) {};
                        \draw (v4) edge (v5) {};
                        \draw (v5) edge (v6) {};
                    \end{tikzpicture}
                \end{subfigure}
                \begin{subfigure}{.48\textwidth}
                    \centering
                    \begin{tikzpicture}[every node/.style={circle, draw, fill=black!10,inner sep=0pt, minimum width=9pt},
                        edge_style/.style={draw=black}]
                        \node (v1) at (0,1) {0};
                    \end{tikzpicture}
                \end{subfigure}
            \end{figure}
        }
        \only<3>{
            \begin{figure}
                \begin{subfigure}{.5\textwidth}
                    \centering
                    \begin{tikzpicture}[every node/.style={circle, draw, fill=black!50,inner sep=0pt, minimum width=4pt},
                        edge_style/.style={draw=black}]
                        \tikzset{red edge/.style={draw=red}}
                        \tikzset{red node/.style={fill=red}}
                        \node (v1)[red node] at (1, 0) {0};
                        \node (v2)[red node] at (0, 1) {1};
                        \node (v3) at (0, 2) {};
                        \node (v4) at (1, 3) {};
                        \node (v5) at (2, 2) {};
                        \node (v6) at (2, 1) {};
                        \draw (v1)[red edge] edge (v2) {};
                        \draw (v1) edge (v5) {};
                        \draw (v1) edge (v4) {};
                        \draw (v1) edge (v3) {};
                        \draw (v1) edge (v6) {};
                        \draw (v2) edge (v3) {};
                        \draw (v2) edge (v4) {};
                        \draw (v2) edge (v5) {};
                        \draw (v3) edge (v4) {};
                        \draw (v4) edge (v5) {};
                        \draw (v5) edge (v6) {};
                    \end{tikzpicture}
                \end{subfigure}
                \begin{subfigure}{.48\textwidth}
                    \centering
                    \begin{tikzpicture}[every node/.style={circle, draw, fill=black!10,inner sep=0pt, minimum width=9pt},
                        edge_style/.style={draw=black}]
                        \node (v1) at (0,0) {0};
                        \node (v2) at (0,1) {1};
                        \draw (v1)[green] edge (v2) {};
                    \end{tikzpicture}
                \end{subfigure}
            \end{figure}
        }
        \only<4>{
            \begin{figure}
                \begin{subfigure}{.5\textwidth}
                    \centering
                    \begin{tikzpicture}[every node/.style={circle, draw, fill=black!50,inner sep=0pt, minimum width=4pt},
                        edge_style/.style={draw=black}]
                        \tikzset{red edge/.style={draw=red}}
                        \tikzset{red node/.style={fill=red}}
                        \node (v1)[red node] at (1, 0) {0};
                        \node (v2)[red node] at (0, 1) {1};
                        \node (v3) at (0, 2) {};
                        \node (v4) at (1, 3) {};
                        \node (v5)[red node] at (2, 2) {2};
                        \node (v6) at (2, 1) {};
                        \draw (v1)[red edge] edge (v2) {};
                        \draw (v1) edge (v5) {};
                        \draw (v1) edge (v4) {};
                        \draw (v1) edge (v3) {};
                        \draw (v1) edge (v6) {};
                        \draw (v2) edge (v3) {};
                        \draw (v2) edge (v4) {};
                        \draw (v2)[red edge] edge (v5) {};
                        \draw (v3) edge (v4) {};
                        \draw (v4) edge (v5) {};
                        \draw (v5) edge (v6) {};
                    \end{tikzpicture}
                \end{subfigure}
                \begin{subfigure}{.48\textwidth}
                    \centering
                    \begin{tikzpicture}[every node/.style={circle, draw, fill=black!10,inner sep=0pt, minimum width=9pt},
                        edge_style/.style={draw=black}]
                        \node (v1) at (0,0) {0};
                        \node (v2) at (0,1) {1};
                        \node (v3) at (0,2) {2};
                        \draw (v1)[green] edge (v2) {};
                        \draw (v2)[red] edge (v3) {};
                    \end{tikzpicture}
                \end{subfigure}
            \end{figure}
        }
        \only<5>{
            \begin{figure}
                \begin{subfigure}{.5\textwidth}
                    \centering
                    \begin{tikzpicture}[every node/.style={circle, draw, fill=black!50,inner sep=0pt, minimum width=4pt},
                        edge_style/.style={draw=black}]
                        \tikzset{red edge/.style={draw=red}}
                        \tikzset{red node/.style={fill=red}}
                        \node (v1)[red node] at (1, 0) {0};
                        \node (v2)[red node] at (0, 1) {1};
                        \node (v3) at (0, 2) {};
                        \node (v4)[red node] at (1, 3) {3};
                        \node (v5)[red node] at (2, 2) {2};
                        \node (v6) at (2, 1) {};
                        \draw (v1)[red edge] edge (v2) {};
                        \draw (v1) edge (v5) {};
                        \draw (v1) edge (v4) {};
                        \draw (v1) edge (v3) {};
                        \draw (v1) edge (v6) {};
                        \draw (v2) edge (v3) {};
                        \draw (v2) edge (v4) {};
                        \draw (v2)[red edge] edge (v5) {};
                        \draw (v3) edge (v4) {};
                        \draw (v4)[red edge] edge (v5) {};
                        \draw (v5) edge (v6) {};
                    \end{tikzpicture}
                \end{subfigure}
                \begin{subfigure}{.48\textwidth}
                    \centering
                    \begin{tikzpicture}[every node/.style={circle, draw, fill=black!10,inner sep=0pt, minimum width=9pt},
                        edge_style/.style={draw=black}]
                        \node (v1) at (0,0) {0};
                        \node (v2) at (0,1) {1};
                        \node (v3) at (0,2) {2};
                        \node (v4) at (0,3) {3};
                        \draw (v1)[green] edge (v2) {};
                        \draw (v2)[red] edge (v3) {};
                        \draw (v3)[blue] edge (v4) {};
                    \end{tikzpicture}
                \end{subfigure}
            \end{figure}
        }
        \only<6>{
            \begin{figure}
                \begin{subfigure}{.5\textwidth}
                    \centering
                    \begin{tikzpicture}[every node/.style={circle, draw, fill=black!50,inner sep=0pt, minimum width=4pt},
                        edge_style/.style={draw=black}]
                        \tikzset{red edge/.style={draw=red}}
                        \tikzset{red node/.style={fill=red}}
                        \node (v1)[red node] at (1, 0) {0};
                        \node (v2)[red node] at (0, 1) {1};
                        \node (v3)[red node] at (0, 2) {4};
                        \node (v4)[red node] at (1, 3) {3};
                        \node (v5)[red node] at (2, 2) {2};
                        \node (v6) at (2, 1) {};
                        \draw (v1)[red edge] edge (v2) {};
                        \draw (v1) edge (v5) {};
                        \draw (v1) edge (v4) {};
                        \draw (v1) edge (v3) {};
                        \draw (v1) edge (v6) {};
                        \draw (v2) edge (v3) {};
                        \draw (v2) edge (v4) {};
                        \draw (v2)[red edge] edge (v5) {};
                        \draw (v3)[red edge] edge (v4) {};
                        \draw (v4)[red edge] edge (v5) {};
                        \draw (v5) edge (v6) {};
                    \end{tikzpicture}
                \end{subfigure}
                \begin{subfigure}{.48\textwidth}
                    \centering
                    \begin{tikzpicture}[every node/.style={circle, draw, fill=black!10,inner sep=0pt, minimum width=9pt},
                        edge_style/.style={draw=black}]
                        \node (v1) at (0,0) {0};
                        \node (v2) at (0,1) {1};
                        \node (v3) at (0,2) {2};
                        \node (v4) at (0,3) {3};
                        \node (v5) at (0,4) {4};
                        \draw (v1)[green] edge (v2) {};
                        \draw (v2)[red] edge (v3) {};
                        \draw (v3)[blue] edge (v4) {};
                        \draw (v4)[yellow] edge (v5) {};
                    \end{tikzpicture}
                \end{subfigure}
            \end{figure}
        }
        \only<7>{
            \begin{figure}
                \begin{subfigure}{.5\textwidth}
                    \centering
                    \begin{tikzpicture}[every node/.style={circle, draw, fill=black!50,inner sep=0pt, minimum width=4pt},
                        edge_style/.style={draw=black}]
                        \tikzset{red edge/.style={draw=red}}
                        \tikzset{red node/.style={fill=red}}
                        \node (v1)[red node] at (1, 0) {0};
                        \node (v2)[red node] at (0, 1) {1};
                        \node (v3)[red node] at (0, 2) {4};
                        \node (v4)[red node] at (1, 3) {3};
                        \node (v5)[red node] at (2, 2) {2};
                        \node (v6)[red node] at (2, 1) {5};
                        \draw (v1)[red edge] edge (v2) {};
                        \draw (v1) edge (v5) {};
                        \draw (v1) edge (v4) {};
                        \draw (v1) edge (v3) {};
                        \draw (v1) edge (v6) {};
                        \draw (v2) edge (v3) {};
                        \draw (v2) edge (v4) {};
                        \draw (v2)[red edge] edge (v5) {};
                        \draw (v3)[red edge] edge (v4) {};
                        \draw (v4)[red edge] edge (v5) {};
                        \draw (v5)[red edge] edge (v6) {};
                    \end{tikzpicture}
                \end{subfigure}
                \begin{subfigure}{.48\textwidth}
                    \centering
                    \begin{tikzpicture}[every node/.style={circle, draw, fill=black!10,inner sep=0pt, minimum width=9pt},
                        edge_style/.style={draw=black}]
                        \node (v1) at (0,0) {0};
                        \node (v2) at (0,1) {1};
                        \node (v3) at (0,2) {2};
                        \node (v4) at (0,3) {3};
                        \node (v5) at (0,4) {4};
                        \node (v6) at (1,3) {5};
                        \draw (v1)[green] edge (v2) {};
                        \draw (v2)[red] edge (v3) {};
                        \draw (v3)[blue] edge (v4) {};
                        \draw (v4)[yellow] edge (v5) {};
                        \draw (v3)[pink] edge (v6) {};
                    \end{tikzpicture}
                \end{subfigure}
            \end{figure}
        }
        \only<8>{
            \begin{figure}
                \begin{subfigure}{.5\textwidth}
                    \centering
                    \begin{tikzpicture}[every node/.style={circle, draw, fill=black!50,inner sep=0pt, minimum width=4pt},
                        edge_style/.style={draw=black}]
                        \tikzset{red edge/.style={draw=red}}
                        \tikzset{green edge/.style={draw=green}}
                        \tikzset{red node/.style={fill=red}}
                        \node (v1)[red node] at (1, 0) {0};
                        \node (v2)[red node] at (0, 1) {1};
                        \node (v3)[red node] at (0, 2) {4};
                        \node (v4)[red node] at (1, 3) {3};
                        \node (v5)[red node] at (2, 2) {2};
                        \node (v6)[red node] at (2, 1) {5};
                        \draw (v1)[red edge] edge (v2) {};
                        \draw (v1)[green edge] edge (v5) {};
                        \draw (v1) edge (v4) {};
                        \draw (v1) edge (v3) {};
                        \draw (v1) edge (v6) {};
                        \draw (v2) edge (v3) {};
                        \draw (v2) edge (v4) {};
                        \draw (v2)[red edge] edge (v5) {};
                        \draw (v3)[red edge] edge (v4) {};
                        \draw (v4)[red edge] edge (v5) {};
                        \draw (v5)[red edge] edge (v6) {};
                    \end{tikzpicture}
                \end{subfigure}
                \begin{subfigure}{.48\textwidth}
                    \centering
                    \begin{tikzpicture}[every node/.style={circle, draw, fill=black!10,inner sep=0pt, minimum width=9pt},
                        edge_style/.style={draw=black}]
                        \node (v1) at (0,0) {0};
                        \node (v2) at (0,1) {1};
                        \node (v3) at (0,2) {2};
                        \node (v4) at (0,3) {3};
                        \node (v5) at (0,4) {4};
                        \node (v6) at (1,3) {5};
                        \draw (v1)[green] edge (v2) {};
                        \draw (v2)[green] edge (v3) {};
                        \draw (v3)[blue] edge (v4) {};
                        \draw (v4)[yellow] edge (v5) {};
                        \draw (v3)[pink] edge (v6) {};
                        \draw (v1)[bend right, green] edge (v3) {};
                    \end{tikzpicture}
                \end{subfigure}
            \end{figure}
        }
        \only<9>{
            \begin{figure}
                \begin{subfigure}{.5\textwidth}
                    \centering
                    \begin{tikzpicture}[every node/.style={circle, draw, fill=black!50,inner sep=0pt, minimum width=4pt},
                        edge_style/.style={draw=black}]
                        \tikzset{red edge/.style={draw=red}}
                        \tikzset{green edge/.style={draw=green}}
                        \tikzset{red node/.style={fill=red}}
                        \node (v1)[red node] at (1, 0) {0};
                        \node (v2)[red node] at (0, 1) {1};
                        \node (v3)[red node] at (0, 2) {4};
                        \node (v4)[red node] at (1, 3) {3};
                        \node (v5)[red node] at (2, 2) {2};
                        \node (v6)[red node] at (2, 1) {5};
                        \draw (v1)[red edge] edge (v2) {};
                        \draw (v1)[green edge] edge (v5) {};
                        \draw (v1)[green edge] edge (v4) {};
                        \draw (v1) edge (v3) {};
                        \draw (v1) edge (v6) {};
                        \draw (v2) edge (v3) {};
                        \draw (v2) edge (v4) {};
                        \draw (v2)[red edge] edge (v5) {};
                        \draw (v3)[red edge] edge (v4) {};
                        \draw (v4)[red edge] edge (v5) {};
                        \draw (v5)[red edge] edge (v6) {};
                    \end{tikzpicture}
                \end{subfigure}
                \begin{subfigure}{.48\textwidth}
                    \centering
                    \begin{tikzpicture}[every node/.style={circle, draw, fill=black!10,inner sep=0pt, minimum width=9pt},
                        edge_style/.style={draw=black}]
                        \node (v1) at (0,0) {0};
                        \node (v2) at (0,1) {1};
                        \node (v3) at (0,2) {2};
                        \node (v4) at (0,3) {3};
                        \node (v5) at (0,4) {4};
                        \node (v6) at (1,3) {5};
                        \draw (v1)[green] edge (v2) {};
                        \draw (v2)[green] edge (v3) {};
                        \draw (v3)[green] edge (v4) {};
                        \draw (v4)[yellow] edge (v5) {};
                        \draw (v3)[pink] edge (v6) {};
                        \draw (v1)[bend right, green] edge (v3) {};
                        \draw (v1)[bend left, green] edge (v4) {};
                    \end{tikzpicture}
                \end{subfigure}
            \end{figure}
        }
        \only<10>{
            \begin{figure}
                \begin{subfigure}{.5\textwidth}
                    \centering
                    \begin{tikzpicture}[every node/.style={circle, draw, fill=black!50,inner sep=0pt, minimum width=4pt},
                        edge_style/.style={draw=black}]
                        \tikzset{red edge/.style={draw=red}}
                        \tikzset{green edge/.style={draw=green}}
                        \tikzset{red node/.style={fill=red}}
                        \node (v1)[red node] at (1, 0) {0};
                        \node (v2)[red node] at (0, 1) {1};
                        \node (v3)[red node] at (0, 2) {4};
                        \node (v4)[red node] at (1, 3) {3};
                        \node (v5)[red node] at (2, 2) {2};
                        \node (v6)[red node] at (2, 1) {5};
                        \draw (v1)[red edge] edge (v2) {};
                        \draw (v1)[green edge] edge (v5) {};
                        \draw (v1)[green edge] edge (v4) {};
                        \draw (v1)[green edge] edge (v3) {};
                        \draw (v1) edge (v6) {};
                        \draw (v2) edge (v3) {};
                        \draw (v2) edge (v4) {};
                        \draw (v2)[red edge] edge (v5) {};
                        \draw (v3)[red edge] edge (v4) {};
                        \draw (v4)[red edge] edge (v5) {};
                        \draw (v5)[red edge] edge (v6) {};
                    \end{tikzpicture}
                \end{subfigure}
                \begin{subfigure}{.48\textwidth}
                    \centering
                    \begin{tikzpicture}[every node/.style={circle, draw, fill=black!10,inner sep=0pt, minimum width=9pt},
                        edge_style/.style={draw=black}]
                        \node (v1) at (0,0) {0};
                        \node (v2) at (0,1) {1};
                        \node (v3) at (0,2) {2};
                        \node (v4) at (0,3) {3};
                        \node (v5) at (0,4) {4};
                        \node (v6) at (1,3) {5};
                        \draw (v1)[green] edge (v2) {};
                        \draw (v2)[green] edge (v3) {};
                        \draw (v3)[green] edge (v4) {};
                        \draw (v4)[green] edge (v5) {};
                        \draw (v3)[pink] edge (v6) {};
                        \draw (v1)[bend right, green] edge (v3) {};
                        \draw (v1)[bend left, green] edge (v4) {};
                        \draw (v1)[bend left, green] edge (v5) {};
                    \end{tikzpicture}
                \end{subfigure}
            \end{figure}
        }
        \only<11>{
            \begin{figure}
                \begin{subfigure}{.5\textwidth}
                    \centering
                    \begin{tikzpicture}[every node/.style={circle, draw, fill=black!50,inner sep=0pt, minimum width=4pt},
                        edge_style/.style={draw=black}]
                        \tikzset{red edge/.style={draw=red}}
                        \tikzset{green edge/.style={draw=green}}
                        \tikzset{red node/.style={fill=red}}
                        \node (v1)[red node] at (1, 0) {0};
                        \node (v2)[red node] at (0, 1) {1};
                        \node (v3)[red node] at (0, 2) {4};
                        \node (v4)[red node] at (1, 3) {3};
                        \node (v5)[red node] at (2, 2) {2};
                        \node (v6)[red node] at (2, 1) {5};
                        \draw (v1)[red edge] edge (v2) {};
                        \draw (v1)[green edge] edge (v5) {};
                        \draw (v1)[green edge] edge (v4) {};
                        \draw (v1)[green edge] edge (v3) {};
                        \draw (v1)[green edge] edge (v6) {};
                        \draw (v2) edge (v3) {};
                        \draw (v2) edge (v4) {};
                        \draw (v2)[red edge] edge (v5) {};
                        \draw (v3)[red edge] edge (v4) {};
                        \draw (v4)[red edge] edge (v5) {};
                        \draw (v5)[red edge] edge (v6) {};
                    \end{tikzpicture}
                \end{subfigure}
                \begin{subfigure}{.48\textwidth}
                    \centering
                    \begin{tikzpicture}[every node/.style={circle, draw, fill=black!10,inner sep=0pt, minimum width=9pt},
                        edge_style/.style={draw=black}]
                        \node (v1) at (0,0) {0};
                        \node (v2) at (0,1) {1};
                        \node (v3) at (0,2) {2};
                        \node (v4) at (0,3) {3};
                        \node (v5) at (0,4) {4};
                        \node (v6) at (1,3) {5};
                        \draw (v1)[green] edge (v2) {};
                        \draw (v2)[green] edge (v3) {};
                        \draw (v3)[green] edge (v4) {};
                        \draw (v4)[green] edge (v5) {};
                        \draw (v3)[green] edge (v6) {};
                        \draw (v1)[bend right, green] edge (v3) {};
                        \draw (v1)[bend left, green] edge (v4) {};
                        \draw (v1)[bend left, green] edge (v5) {};
                        \draw (v1)[bend right, green] edge (v6) {};
                    \end{tikzpicture}
                \end{subfigure}
            \end{figure}
        }
        \only<12>{
            \begin{figure}
                \begin{subfigure}{.5\textwidth}
                    \centering
                    \begin{tikzpicture}[every node/.style={circle, draw, fill=black!50,inner sep=0pt, minimum width=4pt},
                        edge_style/.style={draw=black}]
                        \tikzset{red edge/.style={draw=red}}
                        \tikzset{green edge/.style={draw=green}}
                        \tikzset{red node/.style={fill=red}}
                        \node (v1)[red node] at (1, 0) {0};
                        \node (v2)[red node] at (0, 1) {1};
                        \node (v3)[red node] at (0, 2) {4};
                        \node (v4)[red node] at (1, 3) {3};
                        \node (v5)[red node] at (2, 2) {2};
                        \node (v6)[red node] at (2, 1) {5};
                        \draw (v1)[red edge] edge (v2) {};
                        \draw (v1)[green edge] edge (v5) {};
                        \draw (v1)[green edge] edge (v4) {};
                        \draw (v1)[green edge] edge (v3) {};
                        \draw (v1)[green edge] edge (v6) {};
                        \draw (v2)[green edge] edge (v3) {};
                        \draw (v2) edge (v4) {};
                        \draw (v2)[red edge] edge (v5) {};
                        \draw (v3)[red edge] edge (v4) {};
                        \draw (v4)[red edge] edge (v5) {};
                        \draw (v5)[red edge] edge (v6) {};
                    \end{tikzpicture}
                \end{subfigure}
                \begin{subfigure}{.48\textwidth}
                    \centering
                    \begin{tikzpicture}[every node/.style={circle, draw, fill=black!10,inner sep=0pt, minimum width=9pt},
                        edge_style/.style={draw=black}]
                        \node (v1) at (0,0) {0};
                        \node (v2) at (0,1) {1};
                        \node (v3) at (0,2) {2};
                        \node (v4) at (0,3) {3};
                        \node (v5) at (0,4) {4};
                        \node (v6) at (1,3) {5};
                        \draw (v1)[green] edge (v2) {};
                        \draw (v2)[green] edge (v3) {};
                        \draw (v3)[green] edge (v4) {};
                        \draw (v4)[green] edge (v5) {};
                        \draw (v3)[green] edge (v6) {};
                        \draw (v1)[bend right, green] edge (v3) {};
                        \draw (v1)[bend left, green] edge (v4) {};
                        \draw (v1)[bend left, green] edge (v5) {};
                        \draw (v1)[bend right, green] edge (v6) {};
                        \draw (v2)[bend left, green] edge (v5) {};
                    \end{tikzpicture}
                \end{subfigure}
            \end{figure}
        }
        \only<13>{
            \begin{figure}
                \begin{subfigure}{.5\textwidth}
                    \centering
                    \begin{tikzpicture}[every node/.style={circle, draw, fill=black!50,inner sep=0pt, minimum width=4pt},
                        edge_style/.style={draw=black}]
                        \tikzset{red edge/.style={draw=red}}
                        \tikzset{green edge/.style={draw=green}}
                        \tikzset{red node/.style={fill=red}}
                        \node (v1)[red node] at (1, 0) {0};
                        \node (v2)[red node] at (0, 1) {1};
                        \node (v3)[red node] at (0, 2) {4};
                        \node (v4)[red node] at (1, 3) {3};
                        \node (v5)[red node] at (2, 2) {2};
                        \node (v6)[red node] at (2, 1) {5};
                        \draw (v1)[red edge] edge (v2) {};
                        \draw (v1)[green edge] edge (v5) {};
                        \draw (v1)[green edge] edge (v4) {};
                        \draw (v1)[green edge] edge (v3) {};
                        \draw (v1)[green edge] edge (v6) {};
                        \draw (v2)[green edge] edge (v3) {};
                        \draw (v2)[green edge] edge (v4) {};
                        \draw (v2)[red edge] edge (v5) {};
                        \draw (v3)[red edge] edge (v4) {};
                        \draw (v4)[red edge] edge (v5) {};
                        \draw (v5)[red edge] edge (v6) {};
                    \end{tikzpicture}
                \end{subfigure}
                \begin{subfigure}{.48\textwidth}
                    \centering
                    \begin{tikzpicture}[every node/.style={circle, draw, fill=black!10,inner sep=0pt, minimum width=9pt},
                        edge_style/.style={draw=black}]
                        \node (v1) at (0,0) {0};
                        \node (v2) at (0,1) {1};
                        \node (v3) at (0,2) {2};
                        \node (v4) at (0,3) {3};
                        \node (v5) at (0,4) {4};
                        \node (v6) at (1,3) {5};
                        \draw (v1)[green] edge (v2) {};
                        \draw (v2)[green] edge (v3) {};
                        \draw (v3)[green] edge (v4) {};
                        \draw (v4)[green] edge (v5) {};
                        \draw (v3)[green] edge (v6) {};
                        \draw (v1)[bend right, green] edge (v3) {};
                        \draw (v1)[bend left, green] edge (v4) {};
                        \draw (v1)[bend left, green] edge (v5) {};
                        \draw (v1)[bend right, green] edge (v6) {};
                        \draw (v2)[bend left, green] edge (v5) {};
                        \draw (v2)[bend left, green] edge (v4) {};
                    \end{tikzpicture}
                \end{subfigure}
            \end{figure}
        }
    \end{frame}

    %19
    \begin{frame}
        \frametitle{Außen Aktive Knoten}
        %\begin{Definition}
        %    Betrachten wir einen Knoten v, so heißt ein Nachkomme w von v, außenaktiv, wenn es einen Pfad aus mehreren Nachkommen von w 
        %    gibt ,dessen Knoten nicht mit v und w in einer Bikomponente liegen, und einer Backedge, der zu einem Vorfahren von v führt
        %\end{Definition}
        \centering
        \begin{tikzpicture}[every node/.style={circle, draw, fill=black!10,inner sep=0pt, minimum width=9pt},
            edge_style/.style={draw=black}]
            \tikzset{activ/.style={rectangle}}
            \tikzset{red node/.style={fill=red}}

            \node (v1) at (0,0){0'};
            \node (v2) at (0,1){1};
            \draw (v1) edge (v2){};
            \node (v3)[red node]  at (0,2){1'};
            \node (v4)[activ] at (0,3){2};
            \node (v5) at (1,4){4};
            \draw (v3) edge (v4){};
            \draw (v4)[bend left] edge (v5){};
            \draw (v3) edge (v5){};
            \node (v6) at (0,5){2'};
            \node (v7)[activ] at (0,6){3};
            \draw (v6) edge (v7){};
            \draw (v1) edge[dashed,bend left] (v7){};
        \end{tikzpicture}
        
    
    \end{frame}

    %17
    

    %18
    \begin{frame}
        \frametitle{Zusammenfügen von Bikomponenten}
        \centering
        \only<1>{
            \begin{tikzpicture}[every node/.style={circle, draw, fill=black!10,inner sep=0pt, minimum width=9pt},
                edge_style/.style={draw=black}]
                \tikzset{activ/.style={rectangle}}
                \tikzset{red node/.style={fill=red}}
    
                \node (v1) at (0,0){0};
                \node (v2) at (0,1){1};
                \draw (v1) edge (v2){};
                \node (v3)[red node] at (0,2){1};
                \node (v4)[activ] at (0,3){2};
                \node (v5) at (1,3){3};
                \draw (v3) edge (v4){};
                \draw (v4) edge (v5){};
                \draw (v3) edge (v5){};
                \node (v6) at (0,4){3};
                \node (v7) at (0,5){4};
                \draw (v6) edge (v7){};
                \draw (v3) edge[dashed,bend left, red] (v7){};
                \draw (v4) edge[dashed, bend right] (v1){};

            \end{tikzpicture}
        }
        \only<2>{
            \begin{tikzpicture}[every node/.style={circle, draw, fill=black!10,inner sep=0pt, minimum width=9pt},
                edge_style/.style={draw=black}]
                \tikzset{activ/.style={rectangle}}
                \tikzset{red node/.style={fill=red}}
    
                \node (v1) at (0,0){0};
                \node (v2) at (0,1){1};
                \draw (v1) edge (v2){};
                \node (v3)[red node] at (0,2){1};
                \node (v4)[activ] at (0,3){2};
                \node (v5) at (1,3){3};
                \draw (v3) edge (v4){};
                \draw (v4) edge (v5){};
                \draw (v3) edge (v5){};
                \node (v7) at (0,5){4};
                \draw (v5) edge (v7){};
                \draw (v3) edge[bend left] (v7){};
                \draw (v4) edge[dashed, bend right] (v1){};

            \end{tikzpicture}
        }
        \only<3>{
            \begin{tikzpicture}[every node/.style={circle, draw, fill=black!10,inner sep=0pt, minimum width=9pt},
                edge_style/.style={draw=black}]
                \tikzset{activ/.style={rectangle}}
                \tikzset{red node/.style={fill=red}}
    
                \node (v1) at (0,0){0};
                \node (v2) at (0,1){1};
                \draw (v1) edge (v2){};
                \node (v3)[red node] at (0,2){1};
                \node (v4)[activ] at (0,3){2};
                \node (v5) at (1,3){3};
                \draw (v3) edge (v4){};
                \draw (v4) edge (v5){};
                \draw (v3) edge (v5){};
                \node (v6) at (0,4){3};
                \node (v7) at (0,5){4};
                \draw (v6) edge (v7){};
                \draw (v3) edge[dashed,bend left, red] (v7){};
                \draw (v4) edge[dashed, bend right] (v1){};

            \end{tikzpicture}
        }
        \only<4>{
            \begin{tikzpicture}[every node/.style={circle, draw, fill=black!10,inner sep=0pt, minimum width=9pt},
                edge_style/.style={draw=black}]
                \tikzset{activ/.style={rectangle}}
                \tikzset{red node/.style={fill=red}}
    
                \node (v1) at (0,0){0};
                \node (v2) at (0,1){1};
                \draw (v1) edge (v2){};
                \node (v3)[red node] at (1,2){1};
                \node (v4)[activ] at (1,3){2};
                \node (v5) at (0,3){3};
                \draw (v3) edge (v4){};
                \draw (v4) edge (v5){};
                \draw (v3) edge (v5){};
                \node (v6) at (0,4){3};
                \node (v7) at (0,5){4};
                \draw (v6) edge (v7){};
                \draw (v3) edge[dashed,bend left=70, red] (v7){};
                \draw (v4) edge[dashed, bend left=45] (v1){};

            \end{tikzpicture}
        }
        \only<5>{
            \begin{tikzpicture}[every node/.style={circle, draw, fill=black!10,inner sep=0pt, minimum width=9pt},
                edge_style/.style={draw=black}]
                \tikzset{activ/.style={rectangle}}
                \tikzset{red node/.style={fill=red}}
    
                \node (v1) at (0,0){0};
                \node (v2) at (0,1){1};
                \draw (v1) edge (v2){};
                \node (v3)[red node] at (1,2){1};
                \node (v4)[activ] at (1,3){2};
                \node (v5) at (0,3){3};
                \draw (v3) edge (v4){};
                \draw (v4) edge (v5){};
                \draw (v3) edge (v5){};
                \node (v7) at (0,5){4};
                \draw (v5) edge (v7){};
                \draw (v3) edge[bend left=70] (v7){};
                \draw (v4) edge[dashed, bend left=45] (v1){};

            \end{tikzpicture}
        }
        
    \end{frame}

    %21
    \begin{frame}
        \frametitle{WalkUp}
                Suche einen Pfad von w nach v über die außen liegenden Knoten\\
        \begin{figure}
            \begin{subfigure}{.5\textwidth}
            \visible<2->{    \underbar{Beachte:}\\}
            \only<2>{
                    Außenaktive Knoten dürfen nicht traversiert werden 
            }
            \only<3>{
                  Wird eine Wurzel aus Zwei Außenaktiven Bikomponenten, oder aus zwei Richtungen beschritten, müssen zwei Pfade über die außen liegenden Knoten gefunden werden  
            }
            \only<4>{
                    Eine Außen Aktive Wurzel, die nicht nur durch die Bikomp aus der man kommt außenaktiv ist, darf nur von einer Richtung aus beschritten werden  
            }
                            
        \end{subfigure}     
        \begin{subfigure}{.48\textwidth}
            \centering
            \visible<1-2>{
                \begin{tikzpicture}[every node/.style={circle, draw, fill=black!10,inner sep=0pt, minimum width=9pt},
                    edge_style/.style={draw=black}]
                    \tikzset{activ/.style={rectangle}}
                    \tikzset{red node/.style={fill=red}}
        
                    \node (v1) at (0,0){0};
                    \node (v2) at (0,1){1};
                    \draw (v1) edge (v2){};
                    \node (v3)[red node] at (0,2){1};
                    \node (v4)[activ] at (0,3){2};
                    \node (v5) at (1,3){3};
                    \draw (v3) edge (v4){};
                    \draw (v4) edge (v5){};
                    \draw (v3) edge (v5){};
                    \node (v6) at (0,4){3};
                    \node (v7) at (0,5){4};
                    \draw (v6) edge (v7){};
                    \draw (v3) edge[dashed,bend left, red] (v7){};
                    \draw (v4) edge[dashed, bend right] (v1){};
    
                \end{tikzpicture}
            }
            \only<3>{
                \begin{tikzpicture}[every node/.style={circle, draw, fill=black!10,inner sep=0pt, minimum width=9pt},
                    edge_style/.style={draw=black}]
                    \tikzset{activ/.style={rectangle}}
                    \tikzset{red node/.style={fill=red}}
        
                    \node (v1) at (0,0){0};
                    \node (v2) at (0,1){1};
                    \draw (v1) edge (v2){};
                    \node (v3)[red node] at (0,2){1};
                    \node (v4)[activ] at (0,3){2};
                    \draw (v3) edge (v4){};
                    \node (v6) at (0,4){2};
                    \node (v8) at (1,4){2};
                    \node (v9)[activ] at (1,5){4};
                    \draw (v8) edge (v9){};
                    \node (v7)[activ] at (0,5){3};
                    \draw (v6) edge (v7){};
                    \draw (v3) edge[dashed,bend left, red] (v7){};
                    \draw (v3) edge[dashed,bend right=45, red] (v9){};
                    \draw (v7) edge[bend right, dashed] (v1){};
                    \draw (v1) edge[bend right=41, dashed] (v9){};
    
                \end{tikzpicture}
            }
            \only<4>{
                \begin{tikzpicture}[every node/.style={circle, draw, fill=black!10,inner sep=0pt, minimum width=9pt},
                    edge_style/.style={draw=black}]
                    \tikzset{activ/.style={rectangle}}
                    \tikzset{red node/.style={fill=red}}
        
                    \node (v1) at (0,0){0};
                    \node (v2) at (0,1){1};
                    \draw (v1) edge (v2){};
                    \node (v3)[red node] at (0,2){1};
                    \node (v4)[activ] at (0,3){2};
                    \draw (v3) edge (v4){};
                    \node (v6) at (0,4){2};
                    \node (v8) at (1,4){2};
                    \node (v9)[activ] at (1,5){4};
                    \draw (v8) edge (v9){};
                    \node (v7)[activ] at (0,5){3};
                    \draw (v6) edge (v7){};
                    \draw (v3) edge[dashed,bend left, red] (v7){};
                    \draw (v3) edge[dashed,bend right=45, red] (v9){};
                    \draw (v7) edge[bend right, dashed] (v1){};
                    \draw (v1) edge[bend right=41, dashed] (v9){};
                    \node (v10) at [activ] (1,3){5};
                    \draw (v10) edge [bend left] (v4){};
                    \draw (10) edge [dashed, bend right] (v1){};
    
                \end{tikzpicture}
            }
            \only<5>{
                \begin{tikzpicture}[every node/.style={circle, draw, fill=black!10,inner sep=0pt, minimum width=9pt},
                    edge_style/.style={draw=black}]
                    \tikzset{activ/.style={rectangle}}
                    \tikzset{red node/.style={fill=red}}
        
                    \node (v1) at (0,0){0};
                    \node (v2) at (0,1){1};
                    \draw (v1) edge (v2){};
                    \node (v3)[red node] at (0,2){1};
                    \node (v4)[activ] at (0,3){2};
                    \draw (v3) edge (v4){};
                    \node (v6) at (0,4){2};
                    \node (v8) at (1,4){2};
                    \node (v9)[activ] at (1,5){4};
                    \draw (v8) edge (v9){};
                    \node (v7)[activ] at (0,5){3};
                    \draw (v6) edge (v7){};
                    \draw (v3) edge[dashed,bend left, red] (v7){};
                    \draw (v3) edge[dashed,bend right=45, red] (v9){};
                    \draw (v7) edge[bend right, dashed] (v1){};
                    \draw (v1) edge[bend right=35, dashed] (v9){};
    
                \end{tikzpicture}
            }
        \end{subfigure}      
        \end{figure}
    \end{frame}
    
    %22
    \begin{frame}
        \frametitle{Walk Down}
            Performe DFS auf den Graphen, der durch die Pfade aus dem WalkUp induziert wird\\
        \underbar{Beachte jedoch}\\
        \begin{enumerate}
            \item Füge zuerst die Kante zu einbettung hinzu, die an dem Knoten anliegt, indem du gerade bist
            \item Gibt es eine zuvor besuchte Bikomponte,ohne außen aktive Knoten von der der Momentane Knoten die Wurzel ist, schreite in diese Bikomponente hinein
            \item Gibt es eine zuvor besuchte Bikomponte,mit außen aktive Knoten von der der Momentane Knoten die Wurzel ist, schreite in diese Bikomponente hinein
        \end{enumerate}
        
    
    \end{frame}

    %23
    \begin{frame}
        \frametitle{Kuratowski Minore - Walk Up}
        \centering
            \begin{figure}
                \centering
                \begin{subfigure}{.5\textwidth}
                    \centering
                    \begin{tikzpicture}[every node/.style={circle, draw, fill=black!10,inner sep=0pt, minimum width=9pt},
                        edge_style/.style={draw=black}]
                            \node (v) at (0,0){v};
                            \node (a1)[rectangle] at (-1,0){a};
                            \node (a2)[rectangle] at (1,0){a};
                            \node (r) at (0,1){r};
                            \node (w)[regular polygon,regular polygon sides=5] at (0,2){w};
                            \node (d) at (0,3){d};
                            \draw (v) edge (a1){};
                            \draw (v) edge (a2){};
                            \draw (a1) edge (r){};
                            \draw (a2) edge (r){};
                            \draw (r) edge (w){};
                            \draw (w) edge (d){};
                            \draw (d) edge (a2){};
                            \draw (d) edge (a1){};
                            \draw (w) edge[bend left, dashed] (v){};
                    \end{tikzpicture}
                \end{subfigure}
                \begin{subfigure}{.48\textwidth}
                    \centering
                    \begin{tikzpicture}[every node/.style={circle, draw, fill=black!10,inner sep=0pt, minimum width=9pt,minimum height=9pt},
                        edge_style/.style={draw=black}]
                        \tikzset{activ/.style={rectangle}}
                        \begin{tikzpicture}[every node/.style={circle, draw, fill=black!10,inner sep=0pt, minimum width=9pt, minimum height=9pt},
                            edge_style/.style={draw=black}]
                                \node (a1)[rectangle] at (-1,0){a};
                                \node (v)[rectangle] at (1,0){v};
                                \node (r)[rectangle] at (0,1){r};
                                \node (w)[regular polygon,regular polygon sides=5] at (0,2){w};
                                \node (d) at (0,3){d};
                                \draw (v) edge (a1){};
                                \draw (a1) edge (r){};
                                \draw (a2) edge (r){};
                                \draw (r) edge (w){};
                                \draw (w) edge (d){};
                                \draw (d) edge[bend left] (v){};
                                \draw (d) edge[bend right] (a1){};
                                \draw (w) edge[bend left, dashed] (v){};
                                \draw (d) edge [bend left,] (r){};
                                \draw (w) edge (a1){};
                        \end{tikzpicture}
                    \end{tikzpicture}
                \end{subfigure}
                \end{figure}
    \end{frame}


\end{document}