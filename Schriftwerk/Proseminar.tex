\documentclass[runningheads]{llncs}


\usepackage{subcaption}
\usepackage{graphicx}
\usepackage{tikz}
\usepackage{graphicx}
\usepackage{placeins}
\usepackage[ngerman]{babel}
\usepackage{mathtools}
\usepackage{amsmath}
\usepackage{algorithm2e}

\begin{document}

\title{Planare Graphen - Planare Einbettung}
\author{Manuel Frohn}
\institute{RWTH Aachen University, Aachen, Germany\\
\email{manuel.frohn@rwth-aachen.de}}

\maketitle

\begin{abstract}
\pagenumbering{gobble}  
In dieser Ausarbeitung behandeln wir Planare Graphen. Diese zeichnen sich dadurch aus, dass sie sich in die Ebene einbetten lassen,
ohne dass sich Kanten kreuzen. Im weiteren wird auch ein $\mathcal{O}(n)$ Laufzeit Algorithmus zur Errechnung einer solchen Einbettung vorgestellt.
\keywords{Planare Graphen \and Planare Einbettung}
\end{abstract}

\newpage

\pagenumbering{arabic}
\section{Motivation}
Die Klasse der Planaren Graphen ist sowol für die mathematische Untersuchung von Graphen, als auch zur Modelierung
praktischer Netzwerke, wie zum Beispiel Straßennetze,Platinenschaltkreise oder Landkarten wichtig. So dient die Planarität von Graphen als 
Bedingung für die Funktionalität und/oder Effizienz einiger Algorithmen. Zu diesen zählt zum Beispiel der Algorithmus zur 
Berechnung einer  4-Färbung in polinomieller Zeit.
\\
\\
Die Frage nach einer Planaren Einbettung ist nun eine natürliche Frage. Sie ist zum Beispiel notwendig um ein kurzschlussfreien Platinenschaltkreise zu erstellen.

\section{Einführung}
%planar hervorheben
\begin{definition}[Planarität]
    Ein Graph G heißt planar, wenn man den Graphen so auf eine Ebene zeichnen kann, dass sich seine 
    Kanten nicht schneiden.
\end{definition}
Planare Graphen sind unter anderem durch die sehr graphische Natur ihrere Definition, eine Graphklasse, die schon sehr
lange untersucht wird. Um ein besseres Verständnis für die Graphklasse zu entwickeln wollen wir nun die Abbildungen
aus Fig. 1 betrachten.

\begin{figure}
    \begin{subfigure}{.33\textwidth}
        \centering
        \begin{tikzpicture}[every node/.style={circle, draw, fill=black!50,inner sep=0pt, minimum width=4pt},
            edge_style/.style={draw=black}]
            \node (v1) at (0, 0) {};
            \node (v2) at (1, 0) {};
            \node (v3) at (2, 0) {};
            \node (v4) at (0, 1) {};
            \node (v5) at (1, 1) {};
            \node (v6) at (2, 1) {};  
            \draw (v1) edge (v4);
            \draw (v1) edge (v5);
            \draw (v1) edge (v6);
            \draw (v2) edge (v4);
            \draw (v2) edge (v5);
            \draw (v2) edge (v6);
            \draw (v3) edge (v4);
            \draw (v3) edge (v5);
            \draw (v3) edge (v6);
        \end{tikzpicture}
        \caption{}
    \end{subfigure}
    \begin{subfigure}{.33\textwidth}
        \centering
        \begin{tikzpicture}[every node/.style={circle, draw, fill=black!50,inner sep=0pt, minimum width=4pt},
            edge_style/.style={draw=black}]
            \node (v1) at (1, 0) {};
            \node (v2) at (1, 1) {};
            \node (v3) at (0, 2) {};
            \node (v4) at (2, 2) {};
            \draw (v1) edge (v2);
            \draw (v1) edge (v3);
            \draw (v1) edge (v4);
            \draw (v2) edge (v3);
            \draw (v2) edge (v4);
            \draw (v4) edge (v3);
        \end{tikzpicture}
        \caption{}
    \end{subfigure}
    \begin{subfigure}{.32\textwidth}
        \centering
        \begin{tikzpicture}[every node/.style={circle, draw, fill=black!50,inner sep=0pt, minimum width=4pt},
            edge_style/.style={draw=black}]
            \node (v1) at (0, 0) {};
            \node (v2) at (0, 1) {};
            \node (v3) at (1, 0) {};
            \node (v4) at (1, 1) {};
            \draw (v1) edge (v2);
            \draw (v1) edge (v3);
            \draw (v1) edge (v4);
            \draw (v2) edge (v3);
            \draw (v2) edge (v4);
            \draw (v4) edge (v3);
        \end{tikzpicture}
        \caption{}
    \end{subfigure}
    \caption{Der $ K_{3,3} $ (a) ist nicht planar, der $ K_4 $ (b) jedoch schon, lässt sich aber trozdem mit 
    Kantenüberschneidung zeichnen (c).}
\end{figure}


(a) zeigt den sogenanten $K_{3,3}$, den
volständigen bipariten Graphen, mit jeweils 3 Knoten pro Partition. Dieser ist nicht Planar. Egal wie man den Graphen
zeichnet, man wird nicht in der Lage sein ihn so zu zeichnen, dass sich keine seiner Kanten schneiden. Betrachten wir
nun (b). Wir sehen den $K_4$, den volständigen Graphen, mit 4 Knoten. Dieser ist Planar, was sich in der Abbildung auch
sehr leicht erkennen lässt. Man möchte nun vileicht glauben, dass es sich bei der Planarität um eine Trivial prüfbare
Eigenschaft handelt. Jedoch, wie wir in (c) erkennen können, lässt sich nicht aus jeder Zeichnung eines Planaren Graphen
direkt dessen Planarität erkennen. Bevor wir uns aber damit beschäftigen, wie man, auch algorithmisch, feststellt, ob ein
Graph Planar ist, bedarf es einem kleinem Exkurs. 


\section{Minore}
\begin{definition}[Teilgraph]
    Ein Graph $G' = (V', E') $ heißt Teilgraph von $G = (V, E)$, wenn gilt:
    \[V' \subset V \land \forall e =(a, b) \in E \;mit\;  a,b \in V' \; gilt \; e \in E'.\]
\end{definition}
Ein Teilgraph $G'$ ist also ein Graph, der aus einem Teil der Knoten von G unduziert wird. 
\begin{definition}[Kantenkontraktion]
    Gegeben ein Graph $G = (V,E)$ und eine Kante $ e = {v, w} \in E $, ist das Resultat der Kontraktion von e, der Graph $G' = (V', E')$ mit
    $V' = (V \setminus e) \cup \{n \}$ und $E' = (E \setminus \{v \in e \lor w \in e  \mid e \in E \}) \cup \{ \{n, z\} \mid z,v \in e \in E \lor z,w \in e \in E\}$.
\end{definition}
Die Kontraktions-Operation soll im weiteren betrachtet werden. Schauen wir uns hierzu Abbildung 2 an.

\begin{figure}
    \begin{subfigure}{.25\textwidth}
        \centering
        \begin{tikzpicture}[every node/.style={circle, draw, fill=black!50,inner sep=0pt, minimum width=4pt},
            edge_style/.style={draw=black}]
            \node (v1) at (0, 0) {};
            \node (v2) at (1, 0) {};
            \node (v3) at (2, 0) {};
            \node (v4) at (0, 2) {};
            \node (v5) at (1, 2) {};
            \node (v6) at (2, 2) {};
            \draw (v1) edge (v2) {};
            \draw (v2) edge (v3) {};
            \draw (v2) edge (v5) {};
            \draw (v4) edge (v5) {};
            \draw (v5) edge (v6) {};
        \end{tikzpicture}
    \end{subfigure}
    \begin{subfigure}{.25\textwidth}
        \centering
        \begin{tikzpicture}[every node/.style={circle, draw, fill=black!50,inner sep=0pt, minimum width=4pt},
            edge_style/.style={draw=black}]
            \tikzset{red edge/.style={draw=red}}
            \tikzset{red node/.style={fill=red}}
            \node (v1) at (0, 0) {};
            \node [red node] (v2) at (1, 0) {};
            \node (v3) at (2, 0) {};
            \node (v4) at (0, 2) {};
            \node [red node] (v5) at (1, 2) {};
            \node (v6) at (2, 2) {};
            \draw (v1) edge (v2) {};
            \draw (v2) edge (v3) {};
            \draw[red edge] (v2) edge (v5) {};
            \draw (v4) edge (v5) {};
            \draw (v5) edge (v6) {};
        \end{tikzpicture}
    \end{subfigure}
    \begin{subfigure}{.24\textwidth}
        \centering
        \begin{tikzpicture}[every node/.style={circle, draw, fill=black!50,inner sep=0pt, minimum width=4pt},
            edge_style/.style={draw=black}]
            \tikzset{green node/.style={fill=green}}
            \node (v1) at (0, 0) {};
            \node (v3) at (2, 0) {};
            \node (v4) at (0, 2) {};
            \node (v6) at (2, 2) {};
            \node [green node] (vn) at (1, 1) {};
            \draw (v1) edge (v2) {};
            \draw (v2) edge (v3) {};
            \draw (v4) edge (v5) {};
            \draw (v5) edge (v6) {};
        \end{tikzpicture}
    \end{subfigure}
    \begin{subfigure}{.24\textwidth}
        \centering
        \begin{tikzpicture}[every node/.style={circle, draw, fill=black!50,inner sep=0pt, minimum width=4pt},
            edge_style/.style={draw=black}]
            \tikzset{green edge/.style={draw=green}}
            \node (v1) at (0, 0) {};
            \node (vn) at (1, 1) {};
            \node (v3) at (2, 0) {};
            \node (v4) at (0, 2) {};
            \node (v6) at (2, 2) {};
            \draw [green edge] (v1) edge (vn) {};
            \draw [green edge] (v3) edge (vn) {};
            \draw [green edge] (v4) edge (vn) {};
            \draw [green edge] (v6) edge (vn) {};
        \end{tikzpicture}
    \end{subfigure}
    \caption{Kantenkontraktion an einem Beispiel.}
\end{figure}

Hier sehen wir wie einen Kantenkontraktion funktioniert. Und zwar verschmiltz man die beiden, durch die Kante 
verbundenen Knoten zu Einem. Dazu enfernt man zuerst die zu kontrahierende Kante, und ersetzt die beiden Knoten die sie 
verbindet, durch einen neuen Knoten. Danach ersetzt man jede Kante, die zufor mit einem der beiden verschmolzenen Knoten 
verbunden war, durch eine neue, die nun statdessen mit dem neuen Knoten verbunden ist. Betrachten wir nun das Konzept des
Minors:
\begin{definition}[Minor]
    M heißt Minor von G wenn M aus einem Teilgarphen von G, durch Kantenkontaktion hervorgeht.
\end{definition}
Ein Minor ist nun also das Resultat einer Reduktion eines Teilgraphen.
So ist zum Beispiel der Dreiecksgraph (a) aus Figur 3 ein Minor des Graphen (b)

\begin{figure}
    \begin{subfigure}{.50\textwidth}
        \centering
        \begin{tikzpicture} [every node/.style={circle, draw, fill=black!50,inner sep=0pt, minimum width=4pt},
            edge_style/.style={draw=black}]
            \node (v1) at (0,0) {};
            \node (v2) at (2,0) {};
            \node (v3) at (1,2) {};
            \draw (v1) edge (v2) {};
            \draw (v2) edge (v3) {};
            \draw (v3) edge (v1) {};
        \end{tikzpicture}
        \caption{}
    \end{subfigure}
    \begin{subfigure}{.48\textwidth}
        \centering
        \begin{tikzpicture} [every node/.style={circle, draw, fill=black!50,inner sep=0pt, minimum width=4pt},
            edge_style/.style={draw=black}]
            \node (v1) at (0,0) {};
            \node (v2) at (1,0) {};
            \node (v3) at (2,0) {};
            \node (v4) at (3,0) {};
            \node (v5) at (2,2) {};
            \draw (v1) edge (v2) {};
            \draw (v2) edge (v3) {};
            \draw (v3) edge (v4) {};
            \draw (v4) edge (v5) {};
            \draw (v2) edge (v5) {};
        \end{tikzpicture}
        \caption{}
    \end{subfigure}
    \caption{Minor Beispiel}
\end{figure}
\FloatBarrier
\section{Wichtige Sätze}
Für Planare Graphen gibt es zwei Wichtige Sätze, die wir betrachten wollen.
\begin{theorem}[Eulerscher Polyedersatz]\\
    Für alle zusammenhängende Planare Graphen mit $|V|$ Knoten, $|E|$ Kanten und $|F|$ Flächen gilt:\\
    \[ |V|-|E|+|F|=2 \Leftrightarrow |E| \leq 3|V| - 6 \land |F| \leq 2|V| - 4 \]
\end{theorem}
Beachte das mit Flächen, die von Kanten eingeschlossennen Flächen, plus die sogenanten außere Fläche gemeint ist.
Wie viele es von diesen gibt ist keineswegs einfach zu berechnen, daher wird meisst nur der erste Teil der Umformung
($|E| \leq 3|V| - 6$), als notwendige Bediengung für Planarität, verwendet.

\begin{theorem}[Satz von Kuratowski]\\
    Ein Graph ist genau dan Planar, wenn er den $K_{3,3}$ und den $K_5$ nicht als Minore enthält.  
\end{theorem}
Der Satz von Kuratowski liefert anders als der Eulerscher Polyedersatz eine tatsächliche Grundlage zu Ideentiefizierung
von nicht planaren Graphen. Weiter dient er einiegen Algorithmen, so auch dem der im weiteren Vorgestellt wird als Teil des
Korrektheitsbeweises.
\FloatBarrier
\begin{figure}

    \begin{subfigure}{.50\textwidth}
        \centering
        \begin{tikzpicture}[every node/.style={circle, draw, fill=black!50,inner sep=0pt, minimum width=4pt},
            edge_style/.style={draw=black}]
            \node (v1) at (0, 0) {};
            \node (v2) at (1, 0) {};
            \node (v3) at (2, 0) {};
            \node (v4) at (0, 1) {};
            \node (v5) at (1, 1) {};
            \node (v6) at (2, 1) {};  
            \draw (v1) edge (v4);
            \draw (v1) edge (v5);
            \draw (v1) edge (v6);
            \draw (v2) edge (v4);
            \draw (v2) edge (v5);
            \draw (v2) edge (v6);
            \draw (v3) edge (v4);
            \draw (v3) edge (v5);
            \draw (v3) edge (v6);
        \end{tikzpicture}
        \caption{$K_{3,3}$}
    \end{subfigure}
    \begin{subfigure}{.48\textwidth}
        \centering
        \begin{tikzpicture}[every node/.style={circle, draw, fill=black!50,inner sep=0pt, minimum width=4pt},
            edge_style/.style={draw=black}]
            \node (v1) at (1, 0) {};
            \node (v2) at (2, 0) {};
            \node (v3) at (0, 1) {};
            \node (v4) at (3, 1) {};
            \node (v5) at (2, 2) {};
            \draw (v1) edge (v2);
            \draw (v1) edge (v3);
            \draw (v1) edge (v4);
            \draw (v1) edge (v5);
            \draw (v2) edge (v3);
            \draw (v2) edge (v4);
            \draw (v2) edge (v5);
            \draw (v3) edge (v4);
            \draw (v3) edge (v5);
            \draw (v4) edge (v5);
        \end{tikzpicture}
        \caption{$K_5$}
    \end{subfigure}
    \caption{Kuratowski Minore.}
\end{figure}
\FloatBarrier
\section{Der Einbettungsalgorithmus}
    Der im folgenden vorgestellte Algorithmus, von Myrvold und Boyer, ist in der Lage in $\mathcal{O}(n)$ Laufzeit eine Planare Einbettung
    des Eingabegraphen zu errechnen. Der Algorithmus bettet hierzu Kante für Kante, im äußeren Gebiet ein, bis entweder der gesamte Graph eingebettet ist, oder
    der teils eingebettete Graph ein Kuratowski Minor enthält, in diesem Fall scheitert der Algorithmus und bricht ab.
    
    \subsection{Bikomponenten und Seperatoren}
    Ein wichtiger Teil des Algorithmus sind Bikomponenten. Eine Bikomponente ist ein zusammenhängender maximaler Teilgraph G' von G, ohne Separatoren.
    Ein Separator ist ein Knoten, der bei Entfernung, dazu führen würde, dass der Graph nicht mehr zusammenhängend ist.

    
    \begin{definition}[Separator]
        Gegeben ein Graph $G=(V,E)$ heißt ein Knoten $u \in V$ Separator, wenn für alle Pfade $p = v \xRightarrow{*} w$ mit $v,w \in V$
        gilt: $u \in p$.
    \end{definition}
    \begin{definition}[Bikomponente]
        Ein  maximaler Teilgraph G' von G heißt Bikomponente von G, wenn G' keine Separatoren enthält.
    \end{definition}
    \begin{lemma}
        Ein Graph ist genau dann planar, wenn seine Bikomponenten planar sind.
    \end{lemma}
    
    Dieses Lemma bedeutet, dass der Algorithmus, beim Einbetten einer Kante, nicht den gesamten Graphen G bezüglich potenziellen Kantenüberschneidungen 
    betrachten muss, sondern nur die Bikomponenten, die durch das Hinzufügen, der Kante verändert werden.

    \subsection{Preprocessing}
    Ein wichtiger Schritt, der vor dem Algorithmus, durchgeführt werden muss, 
    ist die Berechnung eines Palmtrees von G. Hierzu berechnen wir zuerst per DepthFirstSearch 
    Den Spanbaum von G und fügen danach die Kanten die nicht wärend des DFS durchlaufen wurden, als BackEdges in den PalmTree ein. 

    \subsection{Der Algortihmus}
    Ein wichtiger Schritt, der vor dem Algorithmus, durchgeführt werden muss, 
    ist die Berechnung eines Palmtrees von G. Hierzu berechnen wir zuerst per DepthFirstSearch 
    den Spanbaum von G und fügen danach die Kanten die nicht wärend des DFS durchlaufen wurden, als BackEdges in den PalmTree ein. 

    Wir können nun damit beginnen den Graphen einzubetten. Wir beginnen, indem wir die Kanten (v,w) des Spannbaums jeweils als eigene Bikomponente einbetten.
    Dabei heißt die Bikomponente die aus (u,v) besteht, wobei u der Spannbaumelternknoten von v ist, Elternbikomponente der Bikomponente die aus der Kante (v,w) besteht.
    Andersherum heißt die Bikomponente aus (v,w) Kindbikomponente, der Bikomponente aus (u,v). Um nun die BackEdges hienzuzufügen, betrachten wir die Knoten in inverser DFI Ordnung und versuchen Kanten (a,b) die den Betrachteten 
    Knoten b mit einem DFS Nochkommen a verbinden, einzubetten. Dabei müssen wir jedoch Externaly Activ Knoten beachten.
    \begin{definition}[Externaly Activ]
        Ein Knoten w aus Bikomponente B, der DFS Nachkomme des zur Zeit bearbeiteten Knotens v ist heißt externaly activ,  
        wenn es einen Pfad aus einer BackEdge und potentiel mehreren Kanten, die nicht in B sind gibt, der zu einem Vorfahren von v führt.
        \\
        Eine Bikomponente heißt externaly activ, wenn sie einen Knoten enthält, der externaly activ ist
    \end{definition}
    Externaly Active sind also all jene Knoten, die später noch in einer Einbettung involviert sein werden. Folglich, nach der Idee des Algorithmus, dürfen
    wir Kanten nur dan hinzufügen, wenn dadurch kein externaly activ Knoten in eine innere Fläche geraten. Dies ereicht man, indem man, bei zum Beispiel einem Externaly Activ
    Knoten auf der rechten Seite der äußeren Fläche, die Kante in die linke Seite einbettet.
    \\
    \\
    Da wir nun wissen, bei welchen Knoten wir mit vorsicht vorgehen müssen, können wir uns nun den restlichen Teil des Algorithmus angucken. Wir fahren fort, indem wir für jede hinzuzufügende Kante
    (a,b) von a ausgehend einen Pfad, in dem teils eingebtteten Graphen nach b suchen. Hierbei ist es aus der Wurzel einer Bikomponente (dem Knoten, mit dem geringsten DFI) die
    zugehörige Elternbikomponente zu betreten. Passiert dies, heißt die Bikomponente aus der man kommt Relevant und wird daher im späteren betrachtet. Entlang des berechneten Pfades werden wir die Kante später einbetten. Die Knoten und Kanten diese Pfades, bis auf maximal die Endpunkte, liegen also nach dem Einbetten der Kante in einer inneren Fläche.
    Daher ergeben sich einige Punkte die es beim suchen eines Pfades zu beachten gilt:
    \begin{enumerate}
        \item Externaly activ Knoten dürfen nicht traversiert werden.
        \item Wird eine Wurzel aus zwei externaly activ Bikomponenten, oder aus zwei Richtungen beschritten,müssen zwei Pfade über die außenliegenden Knoten gefunden werden.
        \item Eine externaly activ Wurzel, die nicht durch die Bikomp aus der man kommt externaly activ ist, darf nur von einer Richtung aus beschritten werden.
    \end{enumerate}

    Nachdem wir dies für alle hinzuzufügenden Kanten gemacht haben, betrachten wir nun den Graphen, der durch die berechneten Pfade induziert wird. Auf diesen wenden wir nun wieder DepthFirstSearch an, bei dem betrachteten Knoten beginnend, mit folgendem Zusatz:
    \begin{enumerate}
        \item Füge zuerst die Kante zu einbettung hinzu, die an dem Knoten anliegt, indem du gerade bist.
        \item Gibt es eine relevante Bikomponte,ohne externaly activ Knoten von der der momentane Knoten die Wurzel ist, schreite in diese Bikomponente hinein.
        \item Gibt es eine relevante Bikomponte,mit außen externaly activ von der der momentane Knoten die Wurzel ist, schreite in diese Bikomponente hinein.

    \end{enumerate}

\end{document}